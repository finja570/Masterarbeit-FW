\chapter{Einleitung}

Diese \LaTeX-Vorlage soll als Leitlinie und Anhaltspunkt für das Schreiben von Abschlussarbeiten dienen. Zum einen zeigt die Vorlage eine typische Struktur einer Abschlussarbeit auf, an der sich viele Arbeiten orientieren. Zum anderen gibt diese Vorlage gleichzeitig Tipps und Hinweise sowohl was die Strukturierung als auch was mögliche Technologien, insbesondere \LaTeX-Pakete, angeht.

Typischer Umfang der Einleitung: 1-3 Seiten.

\section{Motivation}

Der wesentliche Abschnitt der Einleitung ist die Motivation, in der die Arbeit in einen Kontext gesetzt wird. Wichtig ist dabei vor allem, den Leser abzuholen und relativ zügig in das Thema einzuführen. Dazu eigenen sich oft auch Bilder oder Skizzen sehr gut. In vielen Fällen kann dies durch ein konkretes Beispiel gelingen.

Die Motivation darf dabei ruhig auf eine größere Vision hin abzielen, auch wenn im Rahmen der Abschlussarbeit dann vielleicht nur ein bestimmter Aspekt oder eine prototypenhafte Umsetzung realisiert werden kann. Es ist aber wichtig, dem Leser den Gesamtkontext zu vermitteln, damit dann die einzelne Leistung besser eingeordnet werden kann.

„The pursuit of Inclusive Immersion is motivated by four key factors: (i) a moral imperative to make technology advances accessible to all; (ii) VR and AR have demonstrated value as assistive and rehabilitative technologies; (iii) the commercial benefits of reaching the broadest user base possible; and (iv) good design typically yields better usability for all. Expanding on this last factor, there are circumstances in which users may not have a disability but may be situationally impaired. As Wobbrock et al. (2011, pg. 7) observe, “Situational impairments arise when aspects of a user’s environment adversely affect his or her ability to perform certain activities”. A design that addresses a nonsituational impairment may thus also improve the usability for those who are situationally impaired.“ ([Dudley et al., 2023, p. 2990])

Definition motor impairments:
    
„Motor impairment is a loss or limitation of function in muscle control or movement or a limitation in mobility. Common causes include arthritis, paralysis, cerebral palsy, or repetitive strain injury. Motor impairment may also include difficulties in speech control and the need to use input devices other than a mouse or keyboard.“ ([Yuan et al., 2011, p. 83])


\section{Zielsetzung}

Das Ziel der Arbeit ist die Konzeption und prototypische Implementierung von zwei 1-Bit-Interaktionsschnittstellen für das Tool PaneoVR. Dadurch soll Menschen mit motorischen Einschränkungen die Nutzung der Anwendung ermöglicht werden. Dies impliziert, dass sämtliche erforderliche Interaktionen mit dem System durch eine einzige, einfache Aktion, wie das Drücken eines Knopfes oder eine Pust-Interaktion, ausgeführt werden können. Dadurch sollen die mit PaneoVR erstellten Trainings für Menschen mit motorischen Einschränkungen erlebbar werden. Die Generierung von Trainings mit dem Autorenwerkzeug von PaneoVR wird im Rahmen dieser Arbeit zunächst nicht berücksichtigt. Das Ziel besteht in der Gestaltung eines geeigneten Interfaces sowie der systematischen und detaillierten Konzeption der Interaktionen, um eine möglichst barrierearme Nutzung für die Zielgruppe zu gewährleisten. Im Anschluss erfolgt eine umfassende Evaluation der implementierten Interaktionsformen. Im Rahmen dieser Arbeit soll untersucht werden, welche Interaktionsform für den Kontext VR besser geeignet ist und welche Empfehlungen für eine barrierearme Interaktion für Menschen mit motorischen Einschränkungen ausgesprochen werden können. 

\section{Aufbau der Arbeit}

In diesem Abschnitt wird schließlich kurz erklärt, wie der weitere Aufbau der Arbeit ist. Welche Kapitel kommen jetzt noch und mit welchem Thema beschäftigen sich diese? Damit soll dem Leser ein kurzer Überblick gegeben werden. Insbesondere bei einer Bachelorarbeit sollte dieser Abschnitt jedoch sehr knapp gefasst werden.
