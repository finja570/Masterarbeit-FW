\chapter{Einleitung}

Motorische Beeinträchtigungen umfassen den Verlust oder die Einschränkung der Fähigkeit, Muskeln oder Bewegungen zu kontrollieren, und betreffen zahlreiche Menschen weltweit. Die Ursachen sind vielfältig und reichen von Arthritis und Lähmungen bis hin zu neurologischen Erkrankungen wie Zerebralparese oder Verletzungen durch wiederholte Belastung \citep{yuan_game_2011}. Solche Beeinträchtigungen wirken sich nicht nur auf die Mobilität der Personen aus, sondern können auch den Zugang zu digitalen Technologien erschweren. Die Gestaltung barrierefreier Systeme zielt entsprechend darauf ab, Produkte und Anwendungen so zu entwickeln, dass sie von einer möglichst breiten Nutzergruppe mit unterschiedlichen Anforderungen, Fähigkeiten und Fertigkeiten verwendetet werden können. Dies umfasst auch die Berücksichtigung spezifischer Nutzungskontexte, einschließlich der Unterstützung assistiver Technologien \citep{DINISO9241}. 

Virtual Reality (VR) bietet ein wachsendes Spektrum an Anwendungen, von Bildung und Medizin bis hin zu Entertainment und Training. Mit zunehmend erschwinglicher Hardware ist VR nicht mehr nur ein Nischenprodukt, sondern eine Technologie, die für den Consumer-Bereich zunehmend attraktiv wird. Diese Entwicklung eröffnet auch für Menschen mit gesundheitlichen Beeinträchtigungen neue Möglichkeiten. Insbesondere für Personen mit motorischen Beeinträchtigungen könnte VR eine Plattform bieten, die neue Formen der Interaktion und Teilhabe ermöglicht \citep{mott_i_2020}.

In Bezug auf die Barrierefreiheit weist VR jedoch eine gewisse Rückständigkeit im Vergleich zu anderen Technologien auf. Während gängige Betriebssysteme für Smartphones oder Computer bereits umfangreiche Tools zur Barrierefreiheit für Menschen mit unterschiedlichen Beeinträchtigunen integriert haben (vgl. z. B.  \citep{apple_einfuhrung_2024}), fehlen im Bereich VR weithin standardisierte Richtlinien und Werkzeuge \citep{ciccone_next_2023}. Aus dieser Lücke in der Barrierefreiheit resultiert nicht nur eine technische Herausforderung, sondern auch die gesellschaftliche Notwendigkeit, sicherzustellen, dass technologische Fortschritte allen Menschen zugänglich gemacht werden.

\section{Motivation}

Die Entwicklung inklusiver VR-Anwendungen wird durch vier grundlegende Faktoren motiviert. (1) Besonders hervorzuheben ist das ethische und moralische Anliegen, technologische Fortschritte allen Menschen zugänglich zu machen. (2) Darüber hinaus hat sich gezeigt, dass VR-Technologie ein erhebliches Potenzial besitzt, Barrieren in den Bereichen Rehabilitation und Assistenztechnologien abzubauen und somit die Lebensqualität von Menschen zu verbessern. Dies bietet insbesondere für Menschen mit Beeinträchtigungen eine vielversprechende Perspektive. (3) Zusätzlich bringt die Ansprache einer größeren Nutzergruppe auch kommerzielle Vorteile mit sich, da sie den Zugang zu einem breiteren Markt und größeren Absatzchancen ermöglicht. (4) Schließlich führt ein barrierefreies Design häufig zu einer verbesserten Usability für alle Nutzenden. Dieser Aspekt ist besonders relevant im Hinblick auf situationale Einschränkungen, die auftreten, wenn äußere Umstände die Fähigkeit einer Person beeinträchtigen, bestimmte Aktivitäten auszuführen. Auch Personen ohne dauerhafte Beeinträchtigungen können von solchen situationalen Einschränkungen betroffen sein, etwa bei der Nutzung in beengten Räumen oder bei temporären Verletzungen. Ein inklusives Design, das die Bedürfnisse von Menschen mit dauerhaften Einschränkungen berücksichtigt, verbessert somit auch die Nutzungserfahrung für situativ eingeschränkte Personen \citep{dudley_inclusive_2023}.

In der Forschung zur Barrierefreiheit in VR wird zunehmend betont, dass die individuellen Bedürfnisse der Endnutzenden im Zentrum der Entwicklung stehen müssen \citep{dombrowski_designing_2019}. Um für Menschen mit motorischen Beeinträchtigungen dieses Ziel zu erreichen, ist die Entwicklung binärer Interaktionsschnittstellen, die auf einfache Steuerungsmöglichkeiten reduziert sind, ein vielversprechender Ansatz.
Diese erfordern minimale körperliche Anstrengung und ermöglichen damit einer breiten Nutzergruppe den Zugang zu VR-Anwendungen. Solche Schnittstellen bieten nicht nur eine Lösung für Menschen mit motorischen Beeinträchtigungen, sondern verbessern auch die Nutzbarkeit von Systemen in Situationen, in denen physische Bewegungen eingeschränkt oder unpraktisch sind, beispielsweise in beengten Räumen oder bei anderen situationalen Einschränkungen.

\section{Zielsetzung}

Das Ziel dieser Arbeit besteht in der Konzeption und prototypischen Implementierung zweier binärer Interaktionsschnittstellen für die VR-Anwendung PaneoVR. Damit soll Menschen mit motorischen Beeinträchtigungen die Nutzung der Anwendung ermöglicht werden, sodass die mit PaneoVR erstellten VR-Trainings auch für diese Zielgruppe zugänglich und erlebbar werden.

Um eine möglichst barrierearme Nutzung zu gewährleisten, umfasst die Zielsetzung sowohl die Entwicklung eines geeigneten Interfaces als auch die systematische und detaillierte Konzeption der Interaktionen. Eine zentrale Anforderung dabei ist, dass alle erforderlichen Interaktionen mit dem System durch eine binäre Eingabe, bspw. über einen Schalter, ausgeführt werden können. Die implementierten Interaktionsformen sollen anschließend umfassend evaluiert werden und die gewonnenen Erkenntnisse dazu beitragen, Empfehlungen für die Entwicklung barrierearmer VR-Anwendungen abzuleiten.

Zur strukturierten Bearbeitung und Präzisierung dieser Zielsetzung wurden konkret folgende Forschungsfragen definiert, die im Rahmen dieser Arbeit beantwortet werden sollen:

\begin{itemize}
    \item FF1: Wie können binäre Interaktionen in Virtual Reality gestaltet werden, um die Zugänglichkeit für Personen mit motorischen Beeinträchtigungen zu verbessern?
    \item FF2: Welche Herausforderungen treten bei der Implementierung binärer Interaktionsschnittstelle auf und wie können diese überwunden werden?  
    \item FF3: Welche Unterschiede hinsichtlich Faktoren wie Effizienz, Robustheit, Usability und User Experience zeigen sich zwischen den entwickelten Ansätzen? Stimmen diese Ergebnisse mit den Erwartungen aus der Konzeption überein? Ist eine der implementierten Interaktionsformen geeigneter für die Verwendung in VR?
    \item FF4: Welche Empfehlungen können aus der Implementierung und Evaluation der binären Interaktionsschnittstellen in PaneoVR für die Entwicklung barrierearmer VR-Anwendungen abgeleitet werden?
\end{itemize}
 

\section{Aufbau der Arbeit}

Im Verlauf dieser Arbeit werden die aufgestellten Forschungsfragen systematsich in den einzelnen Kapitel adresseirt. Zunächst wird eine inhaltliche Grundlage geschaffen. Hierbei werden relevante theoretische und technologische Konzepte sowie der aktuelle Forschungsstand im Bereich barrierearmer Interaktionsschnittstellen und VR dargestellt (\autoref{chap:2Stand}). Dieser Überblick dient als Basis für die weitere Arbeit und ermöglicht ein umfassendes Verständnis der Thematik.

Im darauffolgenden \autoref{chap:Konzept} wird die Forschungsfrage FF1 addressiert. Hier liegt der Fokus auf der Gestaltung der beiden binären Interaktionsschnittstellen. Zunächst wird ein Design Space definiert, der die Gestaltungsmöglichkeiten und Anforderungen beschreibt. Auf dieser Grundlage werden konkrete Ausprägungen für Interaktionsaufgaben und -komponenten erarbeitet. Diese Ausprägungen werden anschließend systematisch anhand relevanter Parameter bewertet, um die Designentscheidungen nachvollziehbar darzulegen und zu begründen. Abschließend werden die finalen Konzepte für die beiden Interaktionsschnittstellen abgeleitet.

Das \autoref{chap:Implementierung} beschreibt die technische Umsetzung der entwickelten Konzepte. Neben der Vorstellung der Implementierung werden auch die Herausforderungen bei der Realisierung der binären Interaktionsschnittstellen thematisiert (FF2).

Im Anschluss wird im \autoref{chap:Evaluation} zunächst die Konzeption, der Aufbau und die Durchführung der Evaluation vorgestellt. Anschließend werden die Ergebnisse der durchgeführten Evaluation präsentiert.

Diese Ergebnisse werden daraufhin im \autoref{chap:Diskussion} umfassend eingeordnet und interpretiert. Dabei liegt ein besonderer Fokus auf dem Vergleich der beiden Interaktionsschnittstellen hinsichtlich Faktoren wie Effizienz, Robustheit, Usability, User Experience und Motion Sickness. Es wird untersucht, ob die erzielten Ergebnisse mit den im Vorfeld formulierten Erwartungen übereinstimmen und ob eine der entwickelten Interaktionsschnittstellen besser für den Einsatz in VR geeignet ist (FF3). Zudem werden Stärken, Schwächen und mögliche Verbesserungsansätze der Interaktionsschnittstellen erörtert.

Den Abschluss bildet das Kapitel „Fazit und Ausblick“. Hier werden die zentralen Ergebnisse der Arbeit zusammengefasst und Empfehlungen für die zukünftige Entwicklung binärer Interaktionsschnittstellen ausgesprochen. Zudem wird reflektiert, inwieweit die Zielsetzungen der Arbeit erreicht wurden. Abschließend werden mögliche Perspektiven für weiterführende Forschungen aufgezeigt, wobei insbesondere die vierte Forschungsfrage (FF4) adressiert wird.
