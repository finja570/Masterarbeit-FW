\chapter{Fazit}
\label{chap:Zusammenfassung und Ausblick}

In der vorliegenden Arbeit wurde untersucht, wie binäre Interaktionsschnittstellen für VR-Anwendungen gestaltet und im Rahmen des Tools PaneoVR implementiert werden können, um die Zugänglichkeit für Menschen mit motorischen Einschränkungen zu verbessern. Die Untersuchung adressierte systematisch die aufgestellten Forschungsfragen (FF1-FF4), wobei die Konzeption, Implementierung und Evaluation der entwickelten Ansätze zentrale Bestandteile der Analyse bildeten.

Zur Beantwortung der ersten Forschungsfrage (FF1) wurde eine strukturierte Konzeption durchgeführt. Dabei wurde zunächst der Design Space definiert, indem relevante Interaktionsaufgaben und -komponenten sowie Parameter zur Einordnung von Gestaltungsmöglichkeiten identifiziert wurden. Anschließend wurden mögliche Gestaltungsmöglichkeiten detailliert ausgearbeitet und eingeordnet, woraus zwei finale Ansätze abgeleitet wurden. Diese basieren auf den etablierten Scanning-Verfahren Automatic Item Scanning und Continuous Cartesian Scanning, die für den Einsatz in VR angepasst und erweitert wurden.

Die prototypische Implementierung der entwickelten Konzepte ermöglichte die Untersuchung der zweiten Forschungsfrage (FF2). Im Rahmen der Implementierung zeigten sich für beide Ansätze spezifische technische Herausforderungen. Beim Item Scanning stellte insbesondere die dynamische Anpassung des Scannings an unterschiedliche Interaktionsebenen eine Herausforderung dar. Ebenso erforderte die Beschränkung des Scannings auf das Sichtfeld der Nutzenden die Entwicklung eines ressourcenschonenden Verfahrens, um die Performance der VR-Anwendung nicht signifikant zu beeinträchtigen. Das Cartesian Scanning erwies sich als technisch anspruchsvoller, da die korrekte Berechnung der Schnittpunkte der Scan-Linien sowie die Festlegung eines angemessenen Toleranzbereichs für die Interaktion komplex waren. Insgesamt war die Implementierung des Cartesian Scanning aufgrund dieser höheren technischen Anforderungen aufwändiger.

Zur Beantwortung der dritten Forschungsfrage (FF3) - der Untersuchung von Unterschieden zwischen den Ansätzen hinsichtlich Effizienz, Robustheit, Usability und User Experience - wurde eine umfassende Evaluation durchgeführt. Die Ergebnisse zeigen signifikante Unterschiede zwischen den Schnittstellen. Während das Item Scanning hinsichtlich Interaktionsgeschwindigkeit und Robustheit bessere technische Ergebnisse erzielte, überzeugte das Cartesian Scanning in den subjektiven Bewertungen, insbesondere bei Usability und User Experience. Beide Ansätze wiesen spezifische Stärken und Schwächen auf, die sich weitgehend mit den in der Konzeption formulierten Erwartungen deckten. Auffällig war die starke Variation in der Wahrnehmung des Item Scanning zwischen den Testpersonen, wohingegen das Cartesian Scanning eine gleichmäßigere Akzeptanz fand. Zum aktuellen Entwicklungsstand lässt sich festhalten, dass das Cartesian Scanning tendenziell besser angenommen wird, sich jedoch keine eindeutige Präferenz für eine der beiden Schnittstellen hinsichtlich ihrer Eignung für VR-Anwendungen ableiten lässt. Vielmehr zeigen die Ergebnisse, dass beide Ansätze Potenzial für den Einsatz in virtuellen Umgebungen besitzen und es sich lohnt, ihre Weiterentwicklung und Optimierung voranzutreiben.

Die vierte Forschungsfrage (FF4) widmete sich der Ableitung allgemeiner Empfehlungen für die Implementierung binärer Interaktionsschnittstellen in VR-Anwendungen. Die durchgeführte Evaluation lieferte eine fundierte Grundlage für die Formulierung solcher Empfehlungen. Zu den zentralen Empfehlungen gehören die Möglichkeit zur individuellen Anpassung der Scan-Rate, die Implementierung effektiver Mechanismen zur Fehlerkorrektur und die ergonomische Platzierung von UI- und Interaktionselementen.

Eine wesentliche Limitation der durchgeführten Evaluation besteht darin, dass sie ausschließlich mit Testpersonen ohne motorische Beeinträchtigungen durchgeführt wurde. Die Übertragbarkeit der Ergebnisse auf die Zielgruppe ist daher nur eingeschränkt möglich. Zukünftige Arbeiten sollten sich primär der Umsetzung der identifizierten Verbesserungsansätze widmen, insbesondere der Erhöhung der Interaktionsgeschwindigkeit und Robustheit sowie der Reduktion der visuellen Belastung. Darauf aufbauend sollte eine zielgruppenspezifische Evaluation durchgeführt werden, um die Ergebnisse zu validieren und darauf aufbauend weitere Anpassungen vorzunehmen. Dies ist unerlässlich, um die Nutzbarkeit der entwickelten Schnittstellen für die eigentliche Zielgruppe sicherzustellen.

Darüber hinaus ergeben sich aus den Ergebnissen neue Fragestellungen, die in weiterführenden Arbeiten adressiert werden sollten. Dazu gehören die Untersuchung der Einflussfaktoren auf die subjektive Bewertung des Item Scannings sowie eine detaillierte Analyse der Erlernbarkeit beider Schnittstellen. Des Weiteren sollten die Ursachen für das Auftreten von Motion Sickness untersucht und das Evaluationsdesign so angepasst werden, dass externe Einflussfaktoren minimiert werden. Schließlich könnte eine umfassende Evaluation im Hinblick auf Aspekte wie Presence und Immersion weitere Einblicke in die Wechselwirkungen zwischen den Schnittstellen und dem VR-Erlebnis liefern.

Insgesamt zeigt diese Arbeit, dass binäre Interaktionsschnittstellen ein erhebliches Potenzial für barrierearme VR-Anwendungen bieten. Durch gezielte Verbesserungen, zielgruppenspezifische Validierung und iterative Weiterentwicklung könnten die entwickelten Ansätze einen wichtigen Beitrag zur Gestaltung inklusiver virtueller Umgebungen leisten und die Zugänglichkeit für Menschen mit motorischen Beeinträchtigungen nachhaltig verbessern.