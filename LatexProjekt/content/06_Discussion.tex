\chapter{Diskussion}
\label{chap:Diskussion}
%TO DO für dieses Kapitel: Interpretation bei der Effizienz 

In diesem Kapitel werden die Ergebnisse der Evaluation interpretiert und diskutiert, wobei der Fokus darauf liegt, die im Rahmen der Evaluationsplanung formulierten Fragen sowie die Forschungsfrage FF3 zu beantworten. Dies umfasst insbesondere die Untersuchung der Unterschiede zwischen den entwickelten Interaktionsschnittstellen hinsichtlich der Faktoren Effizienz, Robustheit, Usability, User Experience und Motion Sickness. Dabei wird auch geprüft, ob die erzielten Ergebnisse mit den im Vorfeld formulierten Erwartungen übereinstimmen.

Die Erkenntnisse aus der Analyse der Ergebnisse werden anschließend zusammengefasst, indem die Stärken, Schwächen und Verbesserungsmöglichkeiten der entwickelten Ansätze dargestellt werden. Es wird die Frage beantwortet, welches der beiden Verfahren als geeigneter für den Einsatz in VR eingestuft werden kann. Aufbauend auf diesen Erkenntnissen wird diskutert, inwieweit die Ergebnisse dieser Arbeit genutzt werden können, um Empfehlungen für die Entwicklung barrierearmer Interaktionsschnittstellen in VR auszusprechen.
Abschließend wird reflektiert, inwiefern die Zielsetzungen dieser Arbeit erreicht wurden.

\section{Diskussion der Ergebnisse der Evaluation}

\textbf{Effizienz}

Die Effizienz der beiden entwickelten Interaktionsverfahren wurde anhand der Dauer zur Durchführung der Evaluationsabschnitte und der Interaktionsgeschwindigkeit bewertet. Zwischen dem Item Scanning und dem Cartesian Scanning konnten dabei deutliche Unterschiede festgestellt werden.

Zunächst wird betrachtet, wie lang die jeweilige Interaktionsgeschwindigkeit beider Schnittstellen ist und ob eine Schnittstelle eine deutlich schnellere Interaktionsgeschwindigkeit bietet. Außerdem wird betrachtet, wie lange es im Durchschnitt dauert, die Evaluationsabschnitte zu durchlaufen und ob eine Schnittstelle dabei einen schnelleren Durchlauf begünstigt. 
Hinsichtlich der Interaktionsgeschwindigkeiten ist zu erkennen, dass das Item Scanning in beiden Evaluationsabschnitten eine deutlich schnellere mittlere Interaktionsgeschwindigkeit aufweist. Darüber hinaus lassen sich bei beiden Verfahren jeweils Unterschiede zwischen der Interaktionsgeschwindigkeit im technischen und im inhaltsbasierten Abschnitt erkennen. Beim Item Scanning ist dieser Unterschied vor allem auf die geringere Anzahl von Objekten in der Szene zurückzuführen, was dazu führt, dass der Scan schneller das gewünschte Zielobjekt erreicht und dieses somit schneller ausgewählt werden kann. Beim Cartesian Scanning lässt sich die etwas schnelle Geschwindigkeit mit der Position der Objekte im Sichtfeld erklären. Im inhaltsbasierten Abschnitt wurden tendenziell mehr Selektionen mittig im Sichtfeld ausgeführt und vergleichend zum technischen Abschnitt weniger weiter unten im Sichtfeld (vgl. \autoref{fig:bubbleKorrPosGeschwindigkeit}). Dies begünstigt aufgrund der vorliegenden Korrelation zwischen der Interaktionsgeschwindigkeit und der Position des Objekts im Sichtfeld eine schnellere Interaktion. 
Auffallend ist außerdem, dass das Cartesian Scanning eine deutlich größere Spannweite der Interaktionsgeschwindigkeiten aufwies. Dies deutet darauf hin, dass einige Testpersonen mit dem Verfahren vergleichbare Geschwindigkeiten zum Item Scanning erreichen konnten, während andere erheblich länger benötigten. Diese Variabilität kann auf unterschiedliche Nutzungsstrategien zurückgeführt werden, insbesondere auf die gezielte Nutzung von Kopfbewegungen zur Beschleunigung des Scannings. Personen, die ihre Kopfbewegungen bewusst einsetzten, konnten ihre Interaktionen dadurch beschleunigen, während andere ohne diese Strategie langsamer waren. Außerdem variiert die Interaktionsgeschwindigkeit beim Cartesian Scanning stark zwischen Eingaben zur Navigation und Eingaben zur Selektion. Die Navigation wird dabei oft deutlich schneller ausgeführt als die Selektion. Auch dies trägt zu einer höheren Spannweite und Varianz bei. Das Item Scanning hingegen zeigte eine größere Konsistenz zwischen den Testpersonen. Dies liegt insbesondere daran, dass die Schnittstelle den Nutzenden keine Möglichkeit bietet, die Geschwindigkeit des Scannings aktiv zu beeinflussen, vor allem, weil die Nutzenden zwecks der Vergleichbarkeit der Daten nicht die Möglichkeit hatten, die Scan Rate zu verändern. Darüber hinaus ist auch beim Item Scanning erkennbar, dass die Navigation mit schnelleren Interaktionsgeschwindigkeiten einhergeht. So werden bspw. im Navigationsmenü Interaktionen schneller und direkt hintereinander durchgeführt, wenn Nutzende sich mehrfach in eine Richtung drehen möchten. 
Die Unterschiede in der Interaktionsgeschwindigkeit spiegelten sich auch in der Gesamtzeit wider, die die Testpersonen benötigten, um die Evaluationsabschnitte abzuschließen. Für beide Abschnitte benötigten die Teilnehmenden mit dem Cartesian Scanning länger als mit dem Item Scanning, wobei der Unterschied im inhaltsbasierten Szenario besonders deutlich war. Hier waren die Testpersonen mit dem Item Scanning im Mittel über 2 Minuten schneller. Dies unterstreicht, dass das Item Scanning insbesondere bei Szenen mit wenigen Objekten einen Vorteil hinsichtlich der Interaktionsgeschwindigkeit bietet.

Somit ist festzustellen, dass die zuvor formulierten Erwartungen hinsichtlich unterschiedlicher Interaktionsgeschwindigkeiten durch die Evaluationsergebnisse grundsätzlich bestätigt wurden. Allerdings fiel der Unterschied deutlicher aus als angenommen. 

Aus diesen Ergebnissen kann geschlossen werden, dass beim Item Scanning insbesondere die Anzahl der Objekte in der Szene als wichtigster Einflussfaktor für die Interaktionsgeschwindigkeit identifiziert werden konnte. Beim Cartesain Scanning lassen sich zwei relevante Einflussfaktoren identifizieren. Dazu zählt zunächst das Ausmaß, in dem Kopfbewegungen zur Beschleunigung des Scannings eingesetzt werden. Je deutlicher und häufiger diese eingesetzt werden, desto schnellere Interaktionsgeschwindigkeiten können erreicht werden. Der zweite Einflussfaktor ist die Position der Objekte im Sichtfeld. Die durchgeführte Korrelationsanalyse hat gezeigt, dass ein signifikanter Zusammenhang zwischen der Interaktionsgeschwindigkeit und der Position vorliegt. Je weiter oben und links sich ein Objekt im Sichtfeld befindet, desto geringer ist die benötigte Zeit zur Selektion. Dies liegt daran, dass die Scan-Linien entsprechend länger in ihrer Bewegung brauchen, um Objekte weiter rechts oder unten im Sichtfeld zu erreichen. 

Als letztes wurde hinsichtlich der Effizienz untersucht, ob die Interaktionsgeschwindigkeit die wahrgenommene Usability und Effizienz beeinflusst. Für die Usability konnte nur ein schwacher Zusammenhang zwischen der Interaktionsgeschwindigkeit und dem SUS-Score festgestellt werden. Dieser war entsprechend auch statistisch nicht signifikant. Beim Cartesian Scanning zeigte sich eine etwas deutlichere Tendenz, dass längere Interaktionsgeschwindigkeiten mit niedrigeren Usability-Bewertungen in Zusammenhang stehen. Aber auch hier besteht insgesamt nur eine schwache, nicht signifikante Korrelation. 
Hinsichtlich des Zusammenhangs zwischen der Interaktionsgeschwindigkeit und der Bewertung des UEQ-Faktors Effizienz zeigen die Ergebnisse für beide Verfahren unterschiedliche Tendenzen. Während beim Cartesian Scanning nur eine schwache Korrelation bestand, zeigte das Item Scanning eine mäßige, wenn auch statistisch nicht signifikante, Korrelation. Auffällig ist hier, dass es sich um eine positive Korrelation handelt. Das bedeutet, dass langsamere Geschwindigkeiten tendenziell mit besseren Bewertungen der Effizienz einhergehen. %Hier fehlt noch eine Interpretation!! 

\textbf{Robustheit}

Hinsichtlich der Robustheit wurde untersucht, wie häufig Fehler bei der Nutzung der jeweiligen Interaktionsschnittstelle auftreten und was die Gründe für diese Fehler sind. Darüber hinaus wurde analysiert, wie viele Zyklen des Scannings beim Item Scanning durchschnittlich für die Ausführung einer Interaktion nötig waren und wie häufig beim Cartesian Scanning leere Eingaben zur Korrektur von Fehlern eingesetzt wurden. 

Beim Item Scanning traten insgesamt mehr Fehler im technischen Abschnitt auf, in dem nur drei Testpersonen fehlerfrei blieben. Die meisten Testpersonen machten hier einen oder dirket mehrere Fehler, die häufig auf Timing-Probleme zurückzuführen waren. Diese resultierten insbesondere aus der Schwierigkeit, den richtigen Zeitpunkt für die Eingabe zu treffen. Dieses Problem trat bei allen Personen auf, die nicht fehlerfrei blieben. Aus der Rückmeldung der Testpersonen lässt sich schließen, dass u. a. die Animation zwischen den Elementen das Auftreten dieser Fehler begünstigte, da sie zu einer verfrühten Selektion verleitete. Darüber hinaus traten Timing-Probleme verstärkt im Navigationsmenü auf.  Hier kam es oft dazu, dass die Testpersonen eine Navigation durchführten, sich kurz umschauten und im Anschluss eine erneute Navigation in die gleiche Richtung durchführen wollten. Oft wechselte jedoch in diesem Moment der Scan zum nächsten hervorgehobenen Element, wodurch eine Fehlselektion entstand, die eine Navigation in die falsche Richtung auslöste. Mögliche Ansätze, um dem Auftreten dieser Fehler entgegenzuwirken, wären bspw. eine Optimierung der Animation und das Einführen einer längeren Wartezeit auf dem zuvor ausgewählten Objekt im Navigationsmenü. 
Im Rahmen der Gestaltung des technischen Abschnitts für das Item Scanning wurden als Herausforderung eine nicht deutliche Scan-Reihenfolge sowie die Positionierung von Elementen am Rand des Sichtfelds genannt. In der Evaluation traten allerdings keine Fehler auf, die sich auf diese Gründe zurückführen lassen. Längere Wartezeiten bei Zielobjekten, die sich weiter hinten in der Scan-Reihenfolge befinden, haben das Auftreten von Fehlern ebenfalls nicht begünstigt. Auch die Eingabe auf mehreren Ebenen stellte kein Problem dar. 

Beim Cartesian Scanning war das Fehlerprofil vielseitiger. Im technischen Abschnitt blieben mehr Testpersonen fehlerfrei, aber wenn Testpersonen Fehler machten, dann häufig auch direkt zwei oder mehr. Eine der Hauptursachen dafür war die versehentliche Selektion des Hauptmenüs. Wie bereits gezeigt, tendieren die Testpersonen dazu, ihre Blickrichtung so auszurichten, dass die Zielobjekte möglichst mittig im Sichtfeld selektiert werden können. Da das Menü ebenfalls mittig platziert ist, weist es dementsprechend eine räumliche Nähe zu den Zielobjekten auf, was das Auftreten von Selektionsfehlern begünstigt. Außerdem befindet sich das Menü auf der UI-Ebene und damit räumlich näher am Nutzenden. Dadurch wird es von der Implementierung bevorzugt selektiert. Daher können bereits leichte Überdeckungen von Elementen oder Eingaben, die das Zielobjekt in Richtung Menü knapp verfehlen, dazu führen, dass das Menü anstelle des Zielobjekts selektiert wird. Eine andere Positionierung des Menüs könnte somit dem Auftreten dieser Fehlselektionen entgegenwirken. 
Ein weiterer häufiger Grund für Fehler stellt das Vergessen des Wechsels des Interaktionsmodus dar. Testpersonen hatten Schwierigkeiten, sich daran zu erinnern, den Modus zu wechseln, bevor sie mit der Interaktion begannen. So wurde meist ddas Scanning initialisiert und beim Setzen der ersten Linie erkannt, dass es sich um den falschen Modus handelt. Dies zeigt, dass ein Bedarf besteht, den aktuellen Interaktionsmodus auch dann zu visualisieren, wenn das Scanning nicht aktiv ist und die Scan-Linien entsprechend nicht sichtbar sind.
Interessanterweise traten keine Fehler auf, die auf die erwartete Schwierigkeit bei der Selektion von Objekten am Rand des Sichtfelds oder nah aneinander liegender Objekte, wie bspw. bei den Antwortmöglichkeiten von Frage-Feldern, zurückzuführen waren. Hierbei ist anzumerken, dass viele Testpersonen diese Herausforderungen jedoch auch durch Anpassung der Blickrichtung über die Navigation oder die Bewegung des Kopfes umgingen. 

Zusammenfassend bestätigen diese Ergebnisse die im Vorfeld geäußerten Erwartungen hinsichtlich der Unterschiede in der Robustheit der beiden Schnittstellen.

Hinsichtlich der Anzahl der benötigten Zyklen beim Item Scanning zusammengefasst werden, dass die meisten Eingaben im ersten Zyklus erfolgten. Das bedeuetet, Elemente wurden meistens direkt ausgewählt, sobald sie erstmalig hervorgehoben wurden. Dabei traten keine großen Unterschiede zwischen den Evaluationsabschnitten auf. Mehrere Zyklen wurden vor allem bei der Navigation benötigt, da sich die Testpersonen oft zunächst orientieren oder bewusst einen weiteren Zyklus abwarteten, um Timing-Fehler zu vermeiden. Dies zeigt, dass die Möglichkeit, mehrere Zyklen zu durchlaufen von den Testpersonen teilweise als bewusste Strategie zur Fehlervermeidung eingesetzt wurde. Die zuvor beschrieben Optionen zur Optimierung der Selektion im Navigationsmenü könnten daher ebenfalls dazu führen, dass die Notwendigkeit von mehreren Zyklen geringer wird.  

Beim Cartesian Scanning stellte die Nutzung von leeren Eingaben eine Strategie zur Korrektur von Fehlern dar. Im technischen Szenario lag der Mittelwert bei einer leeren Eingabe pro Testperson, während im inhaltsbasierten Szenario ein höherer Mittelwert von 1,75 erreicht wurde. Die Spannweite der Nutzung war dabei jedoch recht groß. Während einige Testpersonen diese Option zur Korrektur gar nicht nutzten, griffen andere häufiger darauf zurück, insbesondere wenn zuvor der Wechsel des Interaktionsmodus vergessen wurde. Bei fehlerhaften Setzen der ersten Scan-Linie wurden leere Eingaben hingegen seltender verwendet. Hier versuchten die meisten Testpersonen eher, dies durch gezielte Kopfbewegungen zu korrigieren, anstatt den Prozess durch eine leere Eingabe abzubrechen und neu zu starten. 
Die Ergebnisse zeigen, dass leere Eingaben ein essenzielles Werkzeug zur Fehlerkorrektur darstellen und die Robustheit der Schnittstelle erhöhen. Ohne diese Option wären vermutlich mehr Fehler aufgetreten. 

\textbf{Erlernbarkeit}

Um zu untersuchen, ob im Verlauf der Evaluation ein messbarer Lerneffekt eingetreten ist, wird analysiert, ob im Verlauf der Evaluation eine Beschleunigung der Interaktionsgeschwindigkeit oder eine Reduktion der Fehleranzahl eingetreten ist.

Beim Item Scanning zeigen die Ergebnisse der durchgeführten linearen Regressionsanalyse, dass die Interaktionsgeschwindigkeit im Verlauf des technischen und im inhaltsbasierten Abschnitt jeweils minimal schneller wird (um 0,839 bzw. 0,391 Sekunden pro Interaktion). Wird die gesamte Evaluation betrachtet, liegt der Effekt bei -2,539 Sekunden und somit deutlich höher. Der Wert wird jedoch durch die geringere Anzahl an Elementen im inhaltsbasierten Abschnitt beeinflusst. Die geringere Anzahl an Elementen hat hier eine geringe Interaktionsgeschwindigkeit begünstigt. Daher ist die Verringerung der Interaktionsgeschwindigkeit über den gesamten Verlauf der Evaluation nicht direkt auf einen Lerneffekt zurückzuführen. Die getrennte Betrachtung der Abschnitte wird daher in diesem Fall als repräsentativer für den Lerneffekt angesehen. Die separate Betrachtung der Entwicklung in den beiden Abschnitten deutet darauf hin, dass keine deutliche Beschleunigung der Interaktionsgeschwindigkeit eingetreten ist, die auf einen Lerneffekt schließen lässt. Die Korrelationskoeffizienten zwischen der Anzahl der getätigten Eingaben und der Interaktionsgeschwindigkeit verdeutlichen dies zusätzlich. Der Zusammenhang ist minimal und im inhaltsbasierten Abschnitt sogar so gering, dass er praktisch keine Bedeutung hat. Dies kann dadurch erklärt werden, dass die Testpersonen beim Item Scanning nicht die Möglichkeit hatten, aktiv Einfluss auf die Geschwindigkeit des Verfahrens zu nehmen, da u. a. die Scan Rate für alle Personen zwecks Vergleichbarkeit der Ergebnisse festgelegt wurde und nicht anpassbar war. Ein Lerneffekt hätte sich theoretisch darüber hinaus in einer geringeren Anzahl an benötigten Scanning-Zyklen zeigen können, was wiederum zu einer geringeren Interaktionsgeschwindigkeit geführt hätte. Da die meisten Eingaben jedoch im ersten oder zweiten Durchgang erfolgen, bleibt auch hier nur minimaler Spielraum für eine Verbesserung.
Beim Cartesian Scanning zeigt sich ein noch schwächerer Zusammenhang zwischen der Anzahl getätigter Eingaben und der Interaktionsgeschwindigkeit. Werden die Abschnitte separat betrachtet, verlangsamt sich die Interaktionsgeschwindigkeit tendenziell sogar im Verlauf des Abschnitts (positiver Regressionskoeffizient). Auf die gesamte Evaluation bezogen ist hingegen eine leichte Beschleunigung (um 0,49 Sekunden) erkennbar. Dieser Wert ist hier aussagekräftiger als beim Item Scanning, da beim Cartesian Scanning Faktoren wie die Anzahl der Elemente in der Szene keinen erkennbaren Einfluss auf die Interaktionsgeschwindigkeit haben. Dennoch bestätigen die Korrelationskoeffizienten, dass auch beim Cartesian Scanning praktisch kein signifikanter Zusammenhang besteht, der das Auftreten eines Lerneffekts aufzeigen könnte.

Ein weiteres Indiz für die Erlernbarkeit eines Verfahrens könnte die Reduktion der Fehlerzahl mit zunehmender Erfahrung sein. Die Ergebnisse zur Robustheit zeigen, dass bei beiden Verfahren die Gesamtzahl der Fehler vom technischen zum inhaltsbasierten Abschnitt abnimmt. Dieser Rückgang könnte auf einen Lerneffekt hindeuten. Die höhere Fehlerzahl im technischen Abschnitt könnte jedoch auch mit der höheren Anzahl an Elementen und der bewusst herausfordernden Gestaltung des Abschnitts zusammenhängen, da diese Faktoren das Auftreten von Fehlern in diesem Abschnitt generell begünstigt haben könnten. Ein Lerneffekt kann daher auch in Bezug auf die Fehlerreduktion anhand der vorliegenden Daten nicht sicher nachgewiesen werden. 
Die Annahme, dass das Item Scanning leichter zu erlernen ist, kann daher anhand der erhobenen objektiven Daten weder bestätigt noch widerlegt werden.

\textbf{Ablenkung durch die Interaktionsschnittstellen}

Um zu untersuchen, inwieweit die Nutzung der Schnittstellen von den Inhalten der VR-Anwendung ablenkt, wurden in Anlehnung an den Presence Questionnaire von \citet{witmer_measuring_1998} drei Fragen formuliert. Diese Fragen sollen Aufschluss darüber geben, ob die Schnittstellen als störend und ablenkend empfunden wurden. 

Die Analyse der Ergebnisse zeigt, dass beide Schnittstellen in diesem Aspekt überwiegend positiv bewertet und als wenig störend wahrgenommen wurden, wobei das Cartesian Scanning tendenziell etwas besser bewertet wurde.
Die Ergebnisse zeigen, dass sich die Testpersonen bei beiden Verfahren gut auf die Inhalte und Aufgaben in der virtuellen Szene konzentrieren konnten. Dies deutet darauf hin, dass die Interaktionsschnittstellen keine übermäßige kognitive Belastung erzeugten. Das Gefühl, dass die Nutzenden in die virtuelle Umgegung involviert sind, war beim Cartesian Scanning durchschnittlich etwas stärker ausgeprägt. Dies könnte auf eine höhere Immersion und ein verstärktes Gefühl von Presence hinweisen. Die Streuung der Bewertungen war hierbei beim Item Scanning höher, was darauf hindeutet, dass die Wahrnehmung der Schnittstelle diesbezüglich stärker von individuellen Präferenzen abhing.

Während beide Schnittstellen es den Testpersonen ermöglichten, sich auf die Inhalte zu fokussieren, gab es beim Cartesian Scanning diesbezüglich größere Unterschiede zwischen den Personen. Dies deutet darauf hin, dass sich einige Testpersonen stärker auf die Bedienung konzentrieren mussten als andere. 

Insgesamt bestätigen die Ergebnisse, dass beide Verfahren so gestaltet sind, dass sie die Aufmerksamkeit der Nutzenden nicht von den Inhalten ablenken und keine übermäßigen Störungen verursachen. Die Annahme, dass die entwickelten Schnittstellen eine geringe Ablenkung vom Inhalt aufweisen, konnte damit bestätigt werden. Diese Ergebnisse zeigen, dass binäre Interaktionsschnittstellen trotz ihrer Einfachheit so gestaltet werden können, dass sie das Erlebnis nicht wesentlich beeinträchtigen und zugleich ein Gefühl von Presence erzeugt werden kann.

\textbf{Usability}

Die Usability der beiden Interaktionsschnittstellen wurde von den Testpersonen unterschiedlich bewertet, wobei das Cartesian Scanning im Vergleich zum Item Scanning deutlich besser abschnitt. Die folgenden Einordnungen der SUS-Scores basieren auf den Ergbnissen von \citet{bangor_empirical_2008}. Der SUS-Score des Cartesian Scanning lag mit einem Durchschnitt von 85,63 im Bereich einer guten bis exzellenten Usability, was etwa einer Schulnote „2“ (amerikanisches B) entspricht. Dies deutet darauf hin, dass die Mehrheit der Testpersonen die Schnittstelle als sehr nutzerfreundlich empfand.
Beim Cartesian Scanning bewerteten 11 von 16 Personen die Usability mit einem Score über 80, was als gut bis exzellent eingeordnet werden kann. Sechs dieser Personen erreichten sogar Werte über 92, die mit einer Schulnote „1“ vergleichbar ist. Lediglich zwei Testpersonen vergaben Scores im Bereich zwischen 50 und 60, was als unterdurchschnittlich betrachtet wird und an der Grenze zur Nicht-Akzeptanz liegt. Insgesamt verdeutlich die Verteilung der Scores, dass die Bewertungen recht einheitlich ausfielen und nur wenige Personen Schwierigkeiten bei der Nutzung der Schnittstelle hatten.

Im Gegensatz dazu erzielte das Item Scanning einen durchschnittlichen SUS-Score von 73,44, der ebenfalls über dem Durchschnitt liegt, aber nur knapp im Bereich einer guten Usability (Schulnote „3“, amerikanisches C). Die Bewertungen zeigen außerdem eine deutlich größere Streuung der Scores. Während neun Personen eine Bewertung über 68 (durchschnittliche Usability) vergaben, bewerteten 7´sieben Personen die Schnittstelle mit einem geringeren Score, wobei vier Testpersonen sogar Scores unter 60 vergaben, die als mangelhaft oder nicht akzeptabel einzustufen sind. Eine Person bewertete die Schnittstelle mit einem Score unter 50, was auf eine klare Nicht-Akzeptanz hinweist.

Diese Verteilung deutet darauf hin, dass die Bewertung des Item Scanning stärker von anderen Einflussfaktoren, wie möglicherweise individuellen Präferenzen, abhängt als das Cartesian Scanning. Während ein Großteil der Testpersonen die Usability als gut bis sehr gut empfand, gab es auch einen deutlichen Anteil, der die Schnittstelle als verbesserungswürdig oder gar nicht akzeptabel bewertete.

Insgesamt zeigt die Analyse, dass das Cartesian Scanning hinsichtlich der Usability als überzeugender bewertet wurde. Beim Item Scanning besteht hingegen noch deutlicher Verbesserungsbedarf, um die Usability für alle Anwender:innen auf ein gleichmäßig hohes Niveau zu bringen. Die Ergebnisse lassen jedoch keine Rückschlüsse darauf zu, welche spezifischen Faktoren beim Item Scanning die Bewertung beeinflusst und damit die hohe Streuung der Ergebnisse hervorgerufen haben. 

\textbf{User Experience}

Die Evaluation der UX zeigt, dass das Cartesian Scanning in den Faktoren durchschnittlich besser bewertet wurde als das Item Scanning, mit Ausnahme des Faktors Originalität. Beide Verfahren wurden in allen UX-Faktoren durchschnittlich positiv bewertet (MW >0,8), wobei sich deutliche Unterschiede in den Einzelbewertungen und damit in den Varianzen zeigen.

Beim Cartesian Scanning wurden vor allem die Faktoren Durchschaubarkeit und Attraktivität positiv hervorgehoben, während die Effizienz am schlechtesten bewertet wurde. Die geringeren Varianzen in den Bewertungen zeigen, dass die Testpersonen sich in ihren Einschätzungen weitgehend einig waren, insbesondere in Bezug auf Durchschaubarkeit und Attraktivität. Items, die besonders positiv wahrgenommenen wurden, waren \textit{verständlich}, \textit{leicht zu lernen}, \textit{angenehm}, \textit{übersichtlich} und \textit{aufgeräumt}. Trotz der allgemein positiven Rückmeldungen wurde das Item \textit{langsam} deutlich negativ bewertet, was die geringe Bewertung des Faktor Effizienz bedingt.
Die UX des Item Scanning wurde ebenfalls positiv bewertet, jedoch mit insgesamt größeren Varianzen, was auf stärkere Uneinigkeit unter den Testpersonen hindeutet. Besonders niedrig bewertet wurden die Faktoren Effizienz und Steuerbarkeit. Die niedrigere Bewertung der Steuerbarkeit könnte u. a. durch die Scan-Reihenfolge bedingt sein. Diese wird durch das System festgelegt und lässt sich nicht aktiv durch den Nutzenden beeinflussen. Insbesondere im technischen Abschnitt empfanden einige Testpersonen die Reihenfolge der Elemente als nicht nachvollziehbar oder nicht erwartungskonform, was aus dem Feedback der Personen hervorgeht. Darüber hinaus könnte auch die geringe Selbstwirksamkeit bzgl. der Interaktionsgeschwindigkeit ein möglicher Grund für die geringere Bewertung des Faktor darstellen. 
Besonders positiv bewertet wurde hingegen der Faktor Originalität. Insbesondere die Items \textit{kreativ} und \textit{neuartig} wurden postiv hervorgehoben. Aber auch das Item \textit{leicht zu lernen} erhilt eine durchaus positive Bewertung. 

Ein zentraler Kritikpunkt bei beiden Schnittstellen ist die Effizienz, die in beiden Fällen am niedrigsten bewertet wurde. Dies ist verständlich, da Scanning-Verfahren grundsätzlich als langsam zu bewerten sind, insbesondere im Vergleich zu anderen gängien Interaktionsmethoden. Eine Verbessung der Bewertung der Effizienz könnte vermutlich bereits durch eine Einstellung zur Anpassung der Scan Rate erreicht werden. Wie die Ergebnisse aus dem Feedback zeigen, wurde diese von vielen Testpersonen als zu langsam empfunden, inbesondere beim Cartesain Scanning. Eine individuelle Einstellung könnte die wahrgenommene Effizienz deutlich beeinflussen. 
Um die Bewertung der Steuerbarkeit des Item Scannings zu verbessern, könnte neben der Anpassbarkeit der Scan Rate auch die Anpassung der Scan-Reihenfolge hilfreich sein. Es wäre denkbar, eine Möglichkeit zu implementieren, die bei Erstellung einer Szene oder im Editor-Modus die Option bietet, die Scan-Reihenfolge festzulegen, damit eine nachvollziehbare Lösung für jedes individuelles Szenario geschaffen werden kann. 

Anschließend zur Auswertung des UEQ wurde ergänzend überprüft, inwieweit die subjektiven Angaben hinsichtlich der Effizienz mit den gemessenen technischen Daten übereinstimmen. Dazu wurde eine Korrelationsanalyse der Bewertung des Faktors Effizenz und der Interaktionsgeschwindigkeit durchgeführt. 
Beim Item Scanning liegt zwischen den Variablen eine mäßige, positive Korrelation vor. Diese zeigt, dass langsamere Interaktionsgeschwindigkeiten tendenziell mit besseren Bewertungen der Effizienz einhergingen. Dieses Ergebnis deutet darauf hin, dass die Bewertung der Effizienz nicht allein durch Geschwindigkeit bestimmt wird und stattdessen viel mehr subjektive Faktoren eine entscheidende Rolle spielen.
Beim Cartesian Scanning wurde nur eine leicht negative Korrelation festgestellt, was bedeutet, dass höhere Geschwindigkeiten tendenziell mit besseren bewertungen der Effizienz zusammenhingen. In diesem Fall scheint die Geschwindigkeit etwas stärker mit der Wahrnehmung von Effizienz verknüpft zu sein als beim Item Scanning. Der Zusammhang ist insgesamt jedoch als recht schwach zu bewerten. Beide Korrelation weisen keine statistische Signifikanz auf, trotzdem verdeutlichen sie, dass eine schnellere Interaktionsgeschwindigkeit kein alleiniger ausschlaggebender Faktor für die Bewertung der Effizienz ist.

\textbf{Motion Sickness}

Die Evaluation der beiden Schnittstellen zeigt, dass Motion Sickness Symptome bei beiden Schnittstellen auftraten, jedoch mit Unterschieden in ihrer Ausprägung und Verteilung. Im Durchschnitt war der SSQ-Score beim Item Scanning leicht höher (12,155) als beim Cartesian Scanning (10,051). Beide Werte weisen auf das Vorhandensein signifikanter Symptome hin, wobei die Verteilung der Scores zwischen den Verfahren variiert. Beim Cartesian Scanning hatten mehr als die Hälfte der Testpersonen (10 Personen) einen Score unter 10, was vernachlässigbaren bis minimalen Symptomen entspricht, während drei Personen einen Score über 20 erreichten, was als besorgniserregend gilt. Beim Item Scanning zeigten hingegen acht Personen einem Score unter 10, während 5 Personen einen Score über 20 aufwiesen. Dies deutet darauf hin, dass insgesamt beim Item Scanning häufiger stärkere Symptome auftraten.

Eine Betrachtung der spezifischen Symptome zeigt, dass in der Kategorie Oculomotor die meisten Symptome auftraten, während in der Kategorie Nausea bei beiden Verfahren die wenigsten Symptome berichtet wurden. Beim Item Scanning berichteten die Testpersonen besonders häufig über angestrengte Augen (11 Personen, davon vier mit moderater Ausprägung) sowie Ermüdung, Kopfdruck und verschwommenes Sehen. Beim Cartesian Scanning traten ebenfalls häufig angestrengte Augen (11 Personen, davon zwei moderat) sowie leichte Ermüdung (acht Personen) auf, wobei Symptome wie Kopfdruck und verschwommenes Sehen weniger verbreitet waren als beim Item Scanning. Die stärkeren Ausprägung oculomotorischer Symptome bei beiden Schnittenstellen deutet darauf hin, dass beide Schnittstellen eine kognitive Belastung hervorrufen. 

In der Kategorie Disorientation wurden bei beiden Schnittstellen die größten Streuungen festgestellt. Während acht Personen beim Item Scanning und neun beim Cartesian Scanning vernachlässigbare Symptome berichteten (Score <5), erreichten einige Testpersonen sehr hohe Scores (>40), was auf erhebliche Einschränlungen hindeutet. Es lässt sich erkennen, dass Disorientation zwar selten, jedoch bei einzelnen Personen besonders ausgeprägt auftreten kann.

Ein entscheidender Einflussfaktor auf die Symptome könnte der Aufbau der Evaluation selbst gewesen sein. Beide Schnittenstellen wurden in einer Sitzung getestet, die 45–60 Minuten dauerte, und erforderten die Erlernung neuer Steuerungen sowie die Bearbeitung vieler Aufgaben. Diese intensive kognitive und visuelle Belastung könnte das Auftreten von oculomotorischen Symptomen wie angestrengte Augen, Ermüdung und Konzentrationsschwierigkeiten begünstigt haben. Dennoch sollte über Optimierungen der Verfahren nachgedacht werden, um die Symptome insgesamt zu reduzieren.

Insgesamt zeigen die Ergebnisse, dass die Annahmen hinsichtlich des Komforts teilweise bestätigt werden können. Die Ergebnisse zeigen, dass insbesondere Symptome auftreten, die auf kognitive oder visuelle Anstrengung zurückzuführen sind. Allerdings sind die Symptome teilweise stärker ausgeprägt als erwartet. Inwieweit dies durch das Evaluationsdesign begünstigt wurde, lässt sich nicht mit Sicherheit beurteilen. Entgegen den Erwartungen zeigen die Ergebnisse zudem einen leichten Unterschied zwischen den Schnittstellen. 

Nach der Auswertung des SSQ wurde untersucht, welche Zusammenhänge zwischen dem Auftreten von Motion Sickness und der Bewertung der Usability und UX sowie der Interaktionsgeschwindigkeit bestehen.
Für die Usability ergab die Korrelationsanalyse beim Item Scanning einen schwachen positiven Zusammenhang. Dies bedeutet, dass höhere SSQ-Scores tendenziell mit einer etwas besseren Bewertung der Berwertung der Usability einhergingen. Beim Cartesian Scanning wurde hingegen ein mäßiger negativer Zusammenhang festgestellt. Hier schienen niedrigere SSQ-Scores zu einer besseren Bewertung der Usability beizutragen. Dies entspricht eher der allgemeinen Erwartung, dass eine Schnittstelle, die weniger physische oder kognitive Belastungen verursacht, auch als einfacher und angenehmer wahrgenommen wird. Diese Korrelation war jedoch nicht statistisch signifikant. Ein wichtiger Aspekt, der bei der Interpretation dieser Ergebnisse berücksichtigt werden muss, ist der Zeitpunkt der Datenerhebung. Die SUS wurde unmittelbar nach dem technischen Abschnitt ausgefüllt, während der SSQ erst am Ende der Evaluation erhoben wurde. Symptome, die erst während des inhaltsbasierten Abschnitts auftraten, bspw. aufgrund der längeren Nutzungszeit oder erhöhter kognitiver Anforderung, konnten die SUS-Bewertung daher nicht beeinflussen. Dies könnte die positive Korrelation beim Item Scanning erklären und darauf hindeuten, dass ggf. beim Item Scanning die Symptome erst im weiterer Verlauf der Evaluation aufgetreten sind oder sich hier stärker ausgeprägt haben.

Hinsichtlich der UX wurden in der Korrelationsanalyse insbesondere die Faktoren Attraktivität und Durchschaubarkeit betrachtet. 
Beim Cartesian Scanning zeigte sich ein leichter negativer Zusammenhang zwischen dem SSQ-Score und der Bewertung der Attraktivität. Dies deutet darauf hin, dass die Schnittstelle eher als angenehm und sympatisch wahrgenommen wird, wenn weniger körperliche Symptome auftreten. Beim Item Scanning hingegen wurde ein schwacher positiver Zusammenhang festgestellt. Dadurch, dass der Zusammenhang bei beiden Schnittstellen eher schwach ist, kann angenommen werden, dass das Auftreten von Motion Sickness nur einen geringen Einfluss auf die Bewertung der Attraktivität hat und hier andere Faktoren eine wichtigere Rolle einnehmen. 

Ein deutlicherer Zusammenhang besteht zwischen dem SSQ und dem UX-Faktor Durchschaubarkeit. Hier wurde bei beiden Schnittstellen eine negative Korrelation festgestellt, die beim Cartesian Scanning als stark bewertet werden kann und sogar eine statistische Signifikanz aufwies. Dies legt nahe, dass weniger Symptome mit einer besseren Bewertung der Durchschaubarkeit einhergehen. Dies deutet darauf hin, dass eine klar strukturierte und leicht verständliche Schnittstelle die Testpersonen weniger belastet und somit weniger Symptome hervorruft. Es wäre aber auch möglich, dass eine Schnittstelle, das weniger Symptome verursacht, wiederum als übersichtlicher und intuitiver empfunden wird. Die Frage nach der Kausalität kann daher an dieser Stelle nicht beantwortet werden. 

Die Korrelationsanalyse zwischen den SSQ-Scores und der Interaktionsgeschwindigkeit zeigt, dass schnellere Interaktionsgeschwindigkeiten tendenziell mit niedrigeren SSQ-Scores und somit weniger Symptomen einhergehen. Beim Cartesian Scanning war dieser Zusammenhang stark und statistisch signifikant, während er beim Item Scanning mäßig ausgeprägt war. Dieser Zusammenhang kann auf zwei Arten interpretiert werden. Einerseits wäre es möglich, dass das Auftreten der Symptome die Interaktionsgeschwindigkeit beeinflusst. Das bedeutet, dass Personen schneller interagieren können, wenn sie weniger Symptome verspüren. Symptome wie angestrengte Augen, Ermüdung oder Konzentrationsschwierigkeiten könnten die Reaktionszeit und somit die Interaktionsgeschwindigkeit beeinträchtigen. Hinsichtlich des Cartesian Scannings wäre es auch möglich, dass Personen, die Symptome verspüren, eher dazu neigen, Kopfbewegungen zu vermeiden, wodurch eine Beschleunigung der Interaktion nicht vorgenommen werden kann. Andererseits könnte auch die Geschwindigkeit die Symptome beeinflussen. Personen, die in der Lage sind, schneller zu interagieren, könnten weniger Symptome entwickeln, da kürzere Interaktionszeiten die okulomotorische und kognitive Belastung verringern könnten.
Beim Item Scanning war die Korrelation schwächer ausgeprägt, was möglicherweise darauf zurückzuführen ist, dass die Interaktionsgeschwindigkeit hier weitgehend durch das System vorgegeben war und nicht aktiv von den Nutzenden beeinflusst werden konnte. Beim Cartesian Scanning hingegen hatten die Testpersonen mehr Kontrolle über die Geschwindigkeit, was zu einer stärkeren Korrelation geführt haben könnte.

\textbf{Limitation der Evaluation hinsichtlich der Testpersonen}

Bei der Einordnung der Ergebnisse ist zu berücksichtigen, dass die Evaluation mit einer vergleichsweise kleinen Anzahl von Testpersonen durchgeführt wurde. Die Ergebnisse können daher vor allem Tendenzen und Hinweise liefern, wobei statistische Signifikanzen aufgrund der begrenzten Datenmenge schwer zu erreichen sind.

Ein weiterer zentraler Aspekt ist die Zielgruppe, auf die die Ergebnisse übertragen werden sollen. Die Testpersonen setzten sich ausschließlich aus Personen ohne motorische Beeinträchtigungen zusammen, die in ihrem Alltag keine binären Interaktionsschnittstellen oder Scanning-Verfahren nutzen. Dies könnte die Ergebnisse in mehrfacher Hinsicht beeinflusst haben. So könnten bspw. Bewertungen im UEQ zu Faktoren wie Originalität oder Stimulation in dieser Evaluation besonders positiv ausgefallen sein, da das Konzept für die Testpersonen eine Neuartigkeit darstellt. Personen aus der Zielgruppe, die in ihrem Alltag bereits Schalter und Scanning-Verfahren verwenden, könnten diese Aspekte hingegen anders bewerten, da ihnen vergleichbare Mechanismen vertrauter sind.
Darüber hinaus könnten sich auch technische Ergebnisse wie Interaktionseschwindigkeiten oder Fehlerraten bei Personen mit motorischen Beeinträchtigungen unterscheiden. Unterschiede in der motorischen Kontrolle oder der Vertrautheit mit binären Schnittstellen könnten dazu führen, dass die Interaktionszeiten und die Präzision der Eingaben abweichend ausfallen. Insgesamt ist daher zu beachten, dass die Ergebnisse der Evaluation in dieser Arbeit nicht ohne Weiteres auf die tatsächliche Zielgruppe übertragbar sind, sondern eine weiterführende Untersuchung mit entsprechenden Testpersonen erforderlich ist, um belastbare und repräsentative Ergebnisse zu erzielen.

\section{Zusammenfassung und Schlussfolgerungen}

Basierend auf den oben dargestellten Ergebnissen werden im Folgenden die Stärken und Schwächen sowie mögliche Verbesserungsansätze der beiden Schnittstellen zusammenfassend dargestellt.

\textbf{Stärken der Schnittstellen}

Das Item Scanning zeichnet sich durch eine schnelle Interaktionsgeschwindigkeit aus, insbesondere wenn die Anzahl der Elemente in der Szene gering ist. Zudem ist die Schnittstelle weniger fehleranfällig und kann insgesamt als robust bezeichnet werden. Die Schnittstelle ermöglicht es den Testpersonen, sich gut auf die Erfüllung ihrer Aufgaben zu konzentrieren, ohne dass die Steuerung als störend oder ablenkend vom Inhalt der virtuellen Umgebung empfunden wird. Die UX des Item Scannings wurde insgesamt positiv bewertet, wobei insbesondere der Faktor Originalität hervorgehoben wurde. Die Schnittstelle wurde als kreativ und innovativ bewertet. Darüber hinaus wird sie von den Testpersonen als verständlich und leicht erlernbar beurteilt. 

Das Cartesian Scanning bietet ein hohes Maß an Kontrolle und Selbstwirksamkeit. Unterstützt wird dies durch die Möglichkeit, die Interaktionsgeschwindigkeit durch Kopfbewegungen zu beschleunigen, sofern diese für die Person realisierbar sind. Darüber hinaus bietet die Schnittstelle effektive Maßnahmen zur Fehlerkorrektur, wie z. B.  die Möglichkeit, leere Eingaben zu tätigen oder eine bereits gesetzte Scan-Linie nachträglich durch Kopfbewegungen zu verschieben.
Wie das Item Scanning lenkt auch das Cartesian Scanning nur wenig vom Inhalt der virtuellen Umgebung ab, sodass die Testpersonen die Aufgaben zielgerichtet bearbeiten konnten. Beim Cartesian Scanning fühlten sich die Nutzenden tendenziell etwas stärker in die Erfahrung involviert als beim Item Scanning.
Sowohl die Usability als auch die UX wurden beim Cartesian Scanning positiv bewertet. In Bezug auf die UX wurde beim Cartesian Scanning insbesondere der Fakotor Durchschaubarkeit positiv bewertet. Die Ergebnisse zeigen, dass die Schnittstelle als angenehm, übersichtlich, leicht erlernbar und aufgeräumt empfunden wurde.

\textbf{Schwächen der Schnittstellen}

Eine zentrale Schwäche des Item Scanning liegt in der geringen Kontrolle der Nutzenden über das Scanning-Verfahren. Weder die Interaktionsgeschwindigkeit noch die Scan-Reihenfolge können aktiv beeinflusst werden, was zu einer geringeren Bewertung der Effizienz und Steuerbarkeit führt. Diese Einschränkungen spiegeln sich in der Bewertung der UX wider, wo die Schnittstelle als langsam, ineffizient und nicht erwartungskonform wahrgenommen wurde. Darüber hinaus wurde die Scan-Reihenfolge als nicht nachvollziehbar empfunden, was zu Verwirrung führte und damit zu der niedrigen Bewertung der Steuerbarkeit beitrug.
Die häufigste Fehlerursache beim Item Scanning waren Timing-Probleme. Während die Robustheit der Schnittstelle an sich positiv bewertet wurde, schränken diese Fehler die Zuverlässigkeit in der Nutzung ein. Auch die Bewertung der Usability zeigte Verbesserungspotential.
Zudem traten beim Item Scanning tendenziell mehr Symptome von Motion Sickness auf. Diese betrafen vor allem angestrengte Augen, Müdigkeit und Kopfdruck, was auf eine mögliche Überforderung der visuellen Wahrnehmung hinweist. Des Weiteren wurde von den Testpersonen bemängelt, dass das Navigationsmenü zu hoch platziert ist, was als störend empfunden wurde. 

Das Cartesian Scanning zeigte insbesondere Schwächen hinsichtlich der Interaktionsgeschwindigkeit und der Robustheit. Die Interaktionsgeschwindigkeit war deutlich langsamer als beim Item Scanning. Zudem besteht eine starke Abhängigkeit zwischen der Position der Zielobjekte im Sichtfeld und der Interaktionsgeschwindigkeit. Diese Aspekte führten zu einer niedrigen Bewertung der Effizienz. Hinsichtlich der Robustheit zeigte sich, dass das Cartesian Scanning im Vergleich zum Item Scanning anfälliger für das Auftreten von Fehlern war und auch die Fehlerursachen vielfältiger waren. 
Auch beim Cartesian Scanning traten Symptome von Motion Sickness auf, insbesondere angestrengte Augen und Ermüdung, was auf eine visuelle Belastung durch das Verfahren hinweist.
Darüber hinaus wurde festgestellt, dass die voreingestellte Scan-Rate etwas zu langsam war, was wahrscheinlich die wahrgenommene Ineffizienz verstärkte. Auch die Zeit, die zum Drücken des Schalters für den Wechsel des Interaktionsmodus benötigt wurde, wurde als zu lang empfunden.

\textbf{Ansätze zur Verbesserung der Schnittstellen}

Für das Item Scanning ist die Optimierung der Animation und der Wartezeiten ein zentraler Ansatz, um Timing-Probleme zu reduzieren und die Robustheit des Verfahrens weiter zu erhöhen. Eine längere Verweildauer auf dem zuvor ausgewählten Element im Navigationsmenü sowie eine etwas längere Wartezeit auf dem ersten Element in der Scan-Reihenfolge könnten die Nutzenden dabei unterstützen, präzisere Selektionen vorzunehmen. Darüber hinaus könnte die Möglichkeit, die Scan-Reihenfolge bereits bei der Erstellung einer Szene festzulegen, die Scan-Reihenfolge nachvollziehbarer erscheinen lassen und damit die Bewertung der Steuerbarkeit verbessern.
Eine weitere Optimierung besteht darin, die Scan Rate konfigurierbar zu machen. Dies würde es den Nutzenden ermöglichen, aktiv Einfluss auf den Prozess und die Interaktionsgeschwindigkeit zu nehmen, was wiederum nicht nur die Effizienz, sondern auch die Steuerbarkeit verbessern könnte. Auch eine neue Positionierung der Menüs könnte die Interaktion erleichtern und die visuelle Belastung reduzieren.

Beim Cartesian Scanning liegt der Schwerpunkt der Verbesserungen insbesondere auf der Erhöhung der Robustheit sowie der Interaktionsgeschwindigkeit und damit der Effizienz. Um die Fehleranfälligkeit zu reduzieren, sollte eine Neupositionierung des Hauptmenüsenüs vorgenommen werden, um Fehlselektionen zu vermeiden. Eine weitere Verbesserungsoption ist die Visualisierung des Interaktionsmodus auch dann, wenn der Scan nicht aktiv ist und die Scan-Linien somit nicht sichtbar sind. 
Um die Interaktionsgeschwindigkeit zu verbessern, sollte auch hier die Möglichkeit geschaffen werden, die Scan Rate individuell anzupassen. Dies könnte sich gleichzeitig positiv auf die Wahrnehmung der Steuerbarkeit auswirken. Eine weitere Idee zur Beschleunigung des Prozesses liegt in der gezielten Anpassung der Scan-Bewegungen. Da die meisten Interaktionen in der Mitte des Sichtfeldes stattfinden, könnte es sinnvoll sein, den Scan so zu gestalten, dass er diese Bereiche schneller erreicht. 

\textbf{Eignung der Schnittstellen für den Einsatz in VR}

Die Analyse der Ergebnisse zeigt, dass sowohl das Item Scanning als auch das Cartesian Scanning Vor- und Nachteile aufweisen und in ihrer aktuellen Implementierung noch Verbesserungspotenzial bieten. Das Cartesian Scanning wurde im Rahmen der Evaluation hinsichtlich der subjektiven Bewertungen in den Fragebögen insgesamt etwas besser und einheitlicher bewertet, während das Item Scanning hinsichtlich der Interaktionsgeschwindigkeit und Fehlerhäufigkeit bessere objektive Ergebnisse aufweist. Die subjektiven Bewertungen beim Item Scanning wiesen zudem häufig größere Streuungen auf. Dies deutet darauf hin, dass die Akzeptanz des Cartesian Scanning im Durchschnitt höher ist, während die Wahrnehmung des Item Scanning stärker von anderen Einflussfaktoren wie etwa persönlichen Präferenzen abhängt.
Trotz der besseren Gesamtbewertung des Cartesian Scanning offenbart insbesondere das freie Feedback, dass einige Testpersonen das Item Scanning deutlich bevorzugten. Auch dies unterstreicht, dass die Wahrnehmung und Akzeptanz der Schnittstellen von persönlichen Präfernezen abhängig ist.
Im derzeitigen Zustand der Implementierung lässt sich daher zwar feststellen, dass das Cartesian Scanning etwas besser angenommen wird, es ergibt sich jedoch keine klare Präferenz für eine der Schnittstellen im Hinblick auf ihre Eignung für den Einsatz in VR. Vielmehr deuten die Ergebnisse darauf hin, dass beide Ansätze Potenzial für den Einsatz in virtuellen Umgebungen bieten und es sich lohnt, ihre Entwicklung und Optimierung weiter voranzutreiben. Durch gezielte Verbesserungen könnten beide Schnittstellen stärker auf die Bedürfnisse unterschiedlicher Nutzenden abgestimmt werden, was ihre Einsatzmöglichkeiten in VR-Anwendungen weiter steigern würde.

\textbf{Ableitung von Empfehlungen für die Entwicklung binärer Interaktionsschnittstellen}

Die durchgeführte Evaluation liefert eine Grundlage, um Empfehlungen für die Entwicklung binärer Interaktionsschnittstellen auszusprechen. Es konnten sowohl Stärken als auch Schwächen der beiden getesteten Schnittstellen identifiziert werden. Darüber hinaus bieten die Ergebnisse durch die Kombination von technischen Daten und subjektivem Nutzerfeedback einen umfassenden Einblick in die Nutzererfahrung und Interaktionsqualität der Schnittstellen. Dennoch müssen die Limitationen der Evaluation berücksichtigt werden. Die Evaluation liefert zwar relevante Hinweise, die Übertragbarkeit auf die Zielgruppe ist jedoch begrenzt. Die Ergebnisse bieten vor allem Ansätze für Empfehlungen, die in weiterführenden zielgruppenspezifischen Untersuchungen validiert und präzisiert werden sollten. 

Aus den Ergebnissen können folgende Empfehlungen abgeleitet werden:

\begin{itemize}
    \item \textbf{Scanning-Verfahren}
    
    Gängige Scanning-Verfahren können auf den Einsatz in VR übertragen werden. Sowohl das Automatic Item Scanning als auch das Continuous Cartesian Scanning eignen sich für den Einsatz in VR. 
    \item \textbf{Anpassbarkeit und Flexibilität}
    
    Es sollte eine Option zur individuellen Anpassung der Scan-Rate angeboten werden. Es ist sinnvoll, verschiedene Scanning-Verfahren zu implementieren, die Nutzende nach ihren individuellen Präferenzen auswählen können. 
    \item \textbf{Klare Visualisierung und Feedback}
    
    Interaktionsprozesse sollten verständlich durch visuelle und akustische Hinweise dargestellt werden, die die Scan-Reihenfolge, aktive Interaktionsmodi oder andere relevante Zustände verdeutlichen.
    \item \textbf{Verständlichkeit}
    
    Bei Verwendung eines Item Scanning-Verfahrens sollte die Scan-Reihenfolge nachvollziehbar und erwartungskonform gestaltet sein. 
    \item \textbf{Fehlerkorrekturmechanismen}
    
    Es sollten Möglichkeiten zur schnellen und intuitiven Fehlerkorrektur implementiert werden, wie z. B. Abbruchfunktionen oder das Zurücksetzen des Scannings, um die Robustheit zu erhöhen.
    \item \textbf{Komfort}
    
    Nicht notwendige visuelle oder kognitive Belastungen sollten so weit wie möglich vermieden werden, um das Auftreten von Motion Sickness Symptomen zu minimieren. 
    \item \textbf{Positionierung von interaktiven Elementen}
    
    UI- und Interaktionselemente sollten so platziert werden, dass sie komfortabel erreichbar sind. Interaktionselemente sollten vorzugsweise in der Mitte des Sichtfeldes platziert werden. 
\end{itemize}

