\chapter{Diskussion}

Hier wird reflektiert, in welchem Umfang die Zielsetzungen der Arbeit erreicht werden konnten. In der Regel gelingt dies durch den Bezug auf die Evaluation.

Typischer Umfang: 1-2 Seiten.

Kapiteleinleinleitung:
-	Folgend werden die Ergebnisse interpretiert und diskutiert
-	Dabei werden die Fragenstellungen, die im Rahmen der Evaluationsplanung formuliert wurden, aufgegriffen und beantwortet 
-	Außerdem wird Bezug zu den Forschungsfragen 3 \& 4 genommen 
-	Es wird geschaut, ob die Ergebnisse mit den Erwartungen übereinstimmen 
-	Anschließend wird darauf eingegangen, inwiefern die Zielsetzungen der Arbeit erfüllt wurden 
Erwartungen:
-	 Item Scanning intuitiver und schneller erlernbar
-	geringere Anzahl von Fehlern bei Item 
-	Item Scanning bei der Navigation langsamer
-	Cartesian steilere Lernkurve
-	Navigation effizienter bei Cartesain
-	Herausforderungen wie eng beieinanderliegende Objekte oder solche am Rand des Sichtfelds könnten zu einer höheren Fehlerquote führen
-	Item Scanning aufgrund der geringen Anzahl von Elementen schneller in inhaltlichen Abschnitt und daher hinsichtlich Effizienz besser bewertet wird
-	Beide Verfahren sollten eine geringe Ablenkung vom Inhalt aufweisen
-	Minimale Entstehung von Motion Sickness 

WAS SONST NOCH FEHLT:
-	Bezug zu Erwartungen aus der Konzeption 
-	SUS 
-	Leere Eingaben Cartesian 
-	Qualitatives Feedback (bei Verbesserungen und Vorteilen aufgreifen) 

Geschwindigkeit und Effizienz
Wie lange dauert es, das Szenario zu durchlaufen? Begünstigt ein Verfahren einen schnelleren Durchlauf? Wie lang ist die Interaktionsgeschwindigkeit bei den Verfahren? Bietet ein Verfahren eine deutlich schnellere Interaktionsgeschwindigkeit?
-	Sowohl in technischen als auch im inhaltsbasierten Szenario brauchten die Testpersonen beim Cartesian Scanning länger 
-	Besonders deutlich ist der Unterschied beim inhaltsbasierten Szenario 
-	Das lässt sich vor allem mit der Zahl der Objekte in der Szene begründen --> beim inhaltsbasierten Szenario waren die meiste Zeit nur wenige Interaktive Elemente in der Szene (ca. 3 Elemente in der Szene \& Navigationsmenu und Hauptmenu). Dadurch erreicht der Scan beim Item Scanning schnell das gewünschte Objekt, wodurch schnell eine Auswahl durchgeführt werden kann. Das spiegelt sich auch in der Interaktionsgeschwindigkeit wider (die ist beim Item Scanning beim inhaltsbasieren Szenario deutlich geringer 2,9 sek. statt 4,6 beim technischen) – entspricht den Erwartungen 
-	Auch beim Cartesain Scanning ist die Interaktionsgeschwindigkeit im inhaltsbasierten Szenario etwas geringer --> Bubbleplot zeigt, dass hier mehr Interaktionen in der Mitte des Sichtfeld durchgeführt wurden als beim technischen und weniger vor allem weiter unten im Sichtfeld. Das begünstigt eine etwas schnellere Interaktionsgeschwindigkeit 
-	Beim Cartesian Scanning ist insbesondere die Spannweite der Geschwindigkeiten deutlich höher als beim Item und auch die Varianz --> es gibt einige Testpersonen, die recht schnell waren und andere die länger brauchten – das kann u.a. mit dem Einsatz von Kopfbewegungen zur Beschleunigung zusammenhängen (es gab Testpersonen, die Kopfbewegungen sehr deutlich eingesetzt haben, um das Verfahren zu beschleunigen, andere haben das nicht oder nur in geringerem Ausmaß gemacht) 
-	D.h. Item Scanning ist zwischen den Personen konstanter was die Interaktionsgeschwindigkeit angeht (bei gleicher Scan Rate ist die Interaktionsgeschwindigkeit ähnlich), Cartesian Scanning hingegen ist weiter gestreut (einige Personen sin damit schnell und vergleichbar zum Item Scanning, andere brauchen deutlich länger)
-	Die Ergebnisse aus der Interaktionsgeschwindigkeit spiegeln sich entsprechend in den Ergebnissen zu Gesamtzeit 

Gibt es Faktoren, die die Interaktionsgeschwindigkeit beeinflussen, etwa die Position der Objekte im Sichtfeld? Gibt es beim Cartesian Scanning deutliche Unterschiede in der Interaktionsgeschwindigkeit abhängig von der Position der Objekte?
-	Die Unterschiede beim Item Scanning zwischen technisch und inhaltsbasiert zeigen, dass hier vor allem die Anzahl der Objekte in der Szene einen Einfluss hat – weniger Objekte = schneller 
-	Beim Cartesian Scanning gibt es eine deutliche Korrelation zwischen der Position des Objekts im Sichtfeld und der Geschwindigkeit --> je weiter sich ein Objekt am oberen oder am linken Rand des Sichtfeld befindet, desto schneller die Interaktionsgeschwindigkeit (das ergibt Sinn, weil natürlich mehr Wartezeit entsteht, wenn die Linien sich durch das Sichtfeld bewegen) 
Führt eine schnellere Interaktionsgeschwindigkeit zu einer besseren Bewertung der Usability und zu einer besseren Bewertung der Effizienz? 
-	Usability: Für beide Verfahren nur ein schwacher Zusammenhang erkennbar (statistisch nicht signifikant) – beim Cartesain Scanning ist es deutlicher als beim Item Scanning, Tendenz dass längere Geschwindigkeiten eher zu niedrigeren SUS-Scores führen 
-	UEQ-Effizienz: nur sehr schwacher Zusammenhang beim Cartesian Scanning, mäßige Korrelation (statistisch jedoch nicht signifikant) beim Item Scanning. Hier aber auffällig, dass es eine positive Korrelation ist, d.h. je schneller die Interaktionsgeschwindigkeit, desto geringer ist tendenziell die Bewertung der Effizienz --> Warum könnte das so sein?
Robustheit
Wie häufig treten Fehler bzw. unbeabsichtigte Eingaben bei den jeweiligen Verfahren auf? 
-	Beim Item Scanning haben im technischen Abschnitt die meisten Personen einen Fehler gemacht, nur 3 keine, Rest mehrere Fehler, im inhaltsbasierten Abschnitt blieben mehr Personen fehlerfrei und keine Person hat mehr als 2 macht 
-	Beim Cartesain haben im technischen Abschnitt mehr Personen keine Fehler gemacht, wenn Fehler passiert sind, dann aber auch meistens direkt 2 o. mehr (nur 3 Personen mit nur einem Fehler), im inhaltsbasierten Abschnitt mehr Personen ohne Fehler und mehr Personen die nur 1 gemacht haben, 3 Personen die mehr als 2 Fehler gemacht haben 
-	Insgesamt sind etwas mehr Fehler beim Cartesian Scanning aufgetreten (entspricht den Erwartungen) 
Was sind die Gründe für auftretende Fehler?
-	Bei Item vor allem zu Beginn das Verständnis (z.b. war manchmal die Aufgabe nicht klar, wodurch Personen mehrfach dasselbe Element nacheinander ausgewählt haben oder das Prinzip des Verfahrens war nicht klar, wodurch andere Elemente gewählt wurden als die Aufgabe vorgesehen hat) 
-	Größtes Problem beim Item Scanning war aber insbesondere das Timing – fast alle Testpersonen (13 von 16) haben Fehler durch zu frühe oder zu späte Eingaben gemacht 
-	Zur Einordung ist das Feedback der Personen hilfreich – einige Beschrieben die Animation als verwirrend, weil diese dazu verleitet hat, die Selektion etwas zu früh durchzuführen \& im Naviagtionsmenü waren Testpersonen manchmal zu langsam (sie haben sich in eine Richtung gedreht, sich umgeschaut \& wollten sich noch einen Schritt weiterdrehen aber dann hat das Scanning zur anderen Richtung gewechselt und es erfolgte eine Rotation in die falsche Richtung) 
-	Ansätze zur Verbesserung: Die Animation optimieren, um Timing-Problemen vorzubeugen und längere Cool Downs im Navigationsmenü 
-	Gründe beim Cartesain Scanning waren vielseitiger 
-	Am häufigsten wurde versehentlich das Hauptmenu ausgewählt, statt es gewünschten Objekts --> es ist im Bubbleplot erkennbar, dass die Personen dazu tendieren, ihre Blickrichtung so auszurichten, dass Zielobjekte recht mittig im Sichtfeld gewählt werden können – da das Menu oben mittig liegt ist es so nah an den anderen Elementen positioniert, was das Auftreten für Verfehlungen erhöht. Außerdem liegt es auf der UI-Ebene und somit räumlich näher am Nutzenden und wird somit durch die Implementierung (die räumlich weitere vorne liegende ebenen präferiert auswählt) bevorzugt. Dadurch können selbst leichte Überschneidungen der Elemente oder Eingaben, die das Zielobjekt knapp Richtung Menu verfehlen dazu führen, dass das Menu gewählt wird  
-	Zweihäufigster Fehler war das Vergessen des Modus Wechsels. Meist wurde der Scan gestartet und beim Setzen der ersten Linie erkannt, dass es sich um den falschen Modus handelt (und die Linien aktuell nicht die gewünschte Farbe haben)  
-	Bei ein paar Personen gab es Probleme beim Schließen der Großansicht der Bilder. Dies liegt wahrscheinlich daran, dass vor Beginn der Evaluation die Blickrichtung in VR nicht zurückgesetzt wurde. Bei Personen, bei denen dies zuvor durchgeführt wurde, traten diese Probleme nicht auf 
-	Ansätze zur Verbesserung: Menu anders positionieren (z.B. weiter am seitlichen Rand), aktuellen Modus noch deutlicher darstellen (auch wenn der Scan gerade nicht läuft) 
-	Es wurde erwartet, dass die Selektion von Objekten am Rand des Sichtfelds (vor allem oben und links) sowie von eng aneinander liegenden Objekten (z.B. bei Antworten von Dialogfeldern) herausfordernd sein könnte und zu Fehlern führen könnte – diese Erwartung ist nicht eingetreten. Keine Testperson machte diesbezüglich Fehler 

Wie häufig werden zwei oder mehr Durchgänge im Scanning benötigt?
-	Recht gleichmäßig beim Item bei inhalt \& technisch (mittelwert 1,5/1,4)
-	Die meisten Eingaben konnten im ersten Zyklus erfolgen (also wurden Objekte meistens beim 1. Mal hervorheben ausgewählt) 
-	Mehrere Zyklen wurden insbesondere bei der Navigation gebraucht (z.B. wenn Personen sich zwischenzeitlich umgesehen haben), andere haben nachdem sie 1-2 Fehler aufgrund des Timings gemacht haben lieber einen Zyklus mehr durchlaufen lassen um Fehler zu vermeiden) oder wenn direkt das erste Objekt in der Scanning Reihenfolge verpasst wurde 
-	Ansätze zur Verbesserung: das erwähnte cool down im Navigationsmenü könnte auch hier Verbesserung bringen, außerdem wäre eine etwas längere Wartezeit beim ersten Objekt hilfreich
Erlernbarkeit
-->	Zeigt sich im Verlauf des Szenarios ein deutlicher (messbarer) Lerneffekt? 

Wie entwickelt sich die Interaktionsgeschwindigkeit im Verlauf des Szenarios? 
-	lineare Regressionsanalyse zeigt, dass in beiden Abschnitten die Interaktionsgeschwindigkeit im Mittel pro Interaktion minimal schneller wird (um 0,839 bzw. 0,391). Wird die gesamte Evaluation betrachtet liegt dieser Wert sogar bei -2,539 – hier ist allerdings zu beachten, dass die Interaktionsgeschwindigkeit im zweiten Abschnitt auf Grund der geringen Anzahl an Elementen geringer ist und somit nicht unbedingt auf die Erlernbarkeit des Verfahrens bezogen werden kann. Dies kann eine Begründung dafür sein, dass der Wert hier deutlich höher ist als bei der getrennten Betrachtung der Abschnitt. Daher kann aus diesem Wert nicht geschlossen werden, dass ein deutlicher Lerneffekt eintrat. Die getrennten Werte sind repräsentativer.
-	Korrelationskoeffizienten verdeutlichen nochmal dass der Zusammenhang zwischen der Anzahl bereits getätigter Eingaben und der Interaktionsgeschwindigkeit nur minimal ist, beim inhaltsbasierten Abschnitt sogar so gering, dass dieser praktisch keine Bedeutung hat 
-	Es kann also festgestellt werden, dass sich anhand der Interaktionsgeschwindigkeit kein Lerneffekt erkennen lässt 
-	Kann damit begründet werden, dass Testpersonen keinen aktiven Einfluss auf die Geschwindigkeit des Scanning nehmen können (weil die Scan Rate zur Vergleichbarkeit einheitlich festgelegt wurde) und somit eine Beschleunigung auch nur in minimalen Umfang möglich ist. Eine Möglichkeit wäre, dass mit steigender Erfahrung die Anzahl benötigter Zyklen und dadurch die Interaktionsgeschwindigkeit niedriger wird. Da aber im Mittel die meisten Eingaben sowohl im technischen als auch im inhaltsbasierten Abschnitt im 1. Oder 2. Durchlauf erfolgen, ist hier eine deutliche Auswirkung oder Verbesserung nicht möglich 
-	Beim Cartesain Scanning ist dies noch geringer 
-	Werden die Abschnitte getrennt betrachtet, verlangsamt sich tendenziell die Geschwindigkeit sogar eher im Verlauf (positiver Regressionskoeffizient)
-	Auf die gesamte Evaluation betrachtet ist jedoch eine leichte Beschleunigung der Interaktionsgeschwindigkeit zu erkennen (um 0,49s)  
-	Hier ist dieser Wert aussagekräftiger, da die Anzahl der Elemente beim Cartesian Scanning keinen direkten Einfluss auf die Interaktionsgeschwindigkeit haben 
-	Die Korrelationskoeffizienten verdeutlichen aber nochmal dass es praktisch kein Zusammenhang zwischen der Anzahl bereits getätigter Eingaben und der Interaktionsgeschwindigkeit vorliegt 
-	Ein Lerneffekt kann hier also auch nicht belegt werden 
Reduziert sich die Anzahl der Fehler mit zunehmender Erfahrung?
-	Um dies zu beantworten werden die Anzahl der gemachten Fehler in beiden Abschnitten genauer betrachtet 
-	Bei beiden Verfahren reduzierte sich die Gesamtanzahl an Fehlern von technischen zum inhaltsbasierten Abschnitt 
-	Es gab jeweils mehr Personen, die den inhaltsbasierten Abschnitt fehlerfrei beendeten als den technischen 
-	Es wurden auch bei beiden Verfahren tendenziell weniger Fehler pro Person gemacht (- mehr Personen die 1-2 Fehler machten und weniger die mehr machten)   
-	Es ist also eine Reduzierung der Fehler erkennbar was auf einen Lerneffekt hindeuten könnte 
-	Allerdings muss betrachtet werden, dass beim technischen Abschnitt für beide Verfahren auch gezielt ein Herausfordernder Aufbau bspw. hinsichtlich der Platzierung der Objekte gewählt wurde, der die Wahrscheinlichkeit des Auftretens von Fehlern im Vergleich zum inhaltsbasierten Abschnitt begünstigen könnte. Es lässt sich also nicht mit Sicherheit sagen, dass die geringe Anzahl an Fehlern durch einen Lerneffekt bedingt wurde
-	Die Erwartung, dass das Cartesian Scanning eine steilere Lernkurve aufweist, kann demnach nicht durch die erhobenen Daten bestätigt werden 

Ablenkung durch die Verfahren
Werden die Interaktionsschnittstellen als störend empfunden? Lenken die Interaktionsschnittstellen die Aufmerksamkeit der Testpersonen vom Inhalt ab?
-	Sehr ähnlich für beide Verfahren bewertet, Cartesian nur leicht besser 
-	Beide überwiegend positiv bewertet 
-	Zeigt, dass die Testpersonen sich bei beiden Verfahren gut auf den Inhalt und die Darstellungen in der Szene konzentrieren konnten 
-	Bei Cartesian haben sich die Personen tendenziell etwas mehr in die Erfahrung involviert gefühlt (also vermutlich ein höheres Gefühl von Presence verspürt) – beim Item Scanning ist hier die Standardabweichung allerdings etwas höher. Das weist darauf hin, dass einige Testpersonen sich sehr involviert gefühlt haben und andere weniger. Beim Cartesian Scanning liegen die Bewertungen näher zusammen
-	Die Personen konnten sich bei beiden Verfahren im Mittel etwa gleich gut auf die Aufgaben und Tätigkeiten in VR konzentrieren und nicht auf die Steuerung – beim Cartesain allerdings eine höhere Standardabweichung – lässt darauf schließen, dass sich einige Testpersonen mehr auf die Steuerung konzentrieren mussten als andere 
-	Insgesamt sehr positive Ergebnisse aus denen sich schließen lässt, dass ein Gefühl von Presence auch beim Einsatz von binären Interaktionsschnittstellen hervorgerufen werden kann und die entwickelten Schnittstellen in der Verwendung nicht zu viel Konzentration verlangen, dass die Steuerung vom Inhalt ablenkt 
-	Erwartung, dass beide Verfahren eine geringe Ablenkung vom Inhalt aufweisen sollten wurde somit bestätigt 

Usability 
ERGÄNZEN!
User Experience
Wie wird die User Experience der Verfahren bewertet? In welchen Faktoren treten (deutliche) Unterschiede in der Bewertung zwischen den Verfahren auf? 
-	Cartesian in 5/6 Faktoren besser bewertet, nur in Originalität schneidet das Item Scanning besser ab 
-	Bei beiden Verfahren wurden alle Faktoren im Mittel positiv bewertet (Mittelwert aller Faktoren >0,8) 
-	Bei Item Effizienz am schlechtesten bewertet gefolgt von Steuerbarkeit 
-	Erwartung, dass das Item Scanning aufgrund der geringen Anzahl von Elementen schneller in inhaltlichen Abschnitt ist und daher hinsichtlich Effizienz besser bewertet wird, wurde also nicht erfüllt 
-	Insgesamt hohe Varianzen bei den Faktoren beim Item Scanning – nur bei Originalität und Durchschaubarkeit etwas geringer – die Testpersonen waren sich in der Bewertung also insgesamt eher uneinig 
-	Items, bei denen der negative Pol am deutlichsten überwiegt: langsam, nicht erwartungskonform, ineffizient, kompliziert 
-	Items, bei denen der positive Pol am deutlichsten überwiegt: verständlich, kreativ, leicht zu lernen, neuartig
-	Bei Cartesian auch Effizienz am schlechtesten bewertet, aber Durchschaubarkeit am besten 
-	Stimulation, Attraktivität und Originalität ähnlich gut bewertet 
-	Insgesamt geringere Varianzen als beim Item Scanning – bedeutet die Testpersonen waren sich hier einiger (insbesondere in Bezug auf die Durchschaubarkeit und Attraktivität, größte Varianz bei der Effizienz) 
-	Items, bei denen der negative Pol am deutlichsten überwiegt: langsam (deutlich weniger als beim Item, insgesamt mehr positive Bewertungen)
-	Items, bei denen der positive Pol am deutlichsten überwiegt: verständlich, leicht zu lernen, neuartig, angenehm, übersichtlich, aufgeräumt
-	Einordnung: Beide Verfahren bei Effizienz am schlechtesten – binäre Interaktionsschnittstellen sind insgesamt langsam (insbesondere im Vergleich zu anderen Interaktionsarten) also verständlich, dass die Testpersonen das so bewerten 
-	Was auch berücksichtigt werden muss – es waren alles Personen ohne motorische Beeinträchtigungen, die in ihrem Alltag keine binären Interaktionsschnittstellen nutzen. Es kann daher sein, dass bspw. die Bewertungen hinsichtlich Originalität besonders positiv ausgefallen sind und Personen die Schalter und Scanning Verfahren auch sonst in ihrem Alltag nutzen dies anders bewerten würden 
-	Steuerbarkeit bei Item: Es besteht keine Möglichkeit, das Scanning zu beeinflussen (Scan Reihenfolge ist durch das System festgelegt), außerdem durch Feedback die Kritik, dass die Reihenfolge nicht immer nachvollziehbar war/den Erwartungen entsprochen hat – daher vermutlich die schlechtere Bewertung (auch hinsichtlich Erwartskonformität)
-	Bei Carteisan hingegen haben die Testpersonen im Feedback angegeben, dass sie in der Nutzung mehr Kontrolle verspürt haben 
-	Die sehr positive Bewertung hinsichtlich Durchschaubarkeit lässt darauf schließen, dass die Personen das Cartesain Scanning insgesamt als leichter und verständlicher empfunden haben 
-	Für beide Verfahren wurde das Item „leicht zu lernen“ besonders positiv bewertet 
-	Insgesamt positive Evaluation, aber aufgrund der geringen Anzahl an Testpersonen gibt es hohe Varianzen, wodurch die Interpretationen nicht allzu zuverlässig sind. Die machen aber Tendenzen deutlich (was wurde besonders positiv wahrgenommen und wo besteht Verbesserungsbedarf) 

Decken sind die subjektiven Angaben hinsichtlich der Effizienz mit den gemessenen technischen Daten?
-	Korrelationsanalyse zeigt einen mäßigen Zusammenhang zwischen der mittleren Interaktionsgeschwindigkeit und der Bewertung der Effizienz beim Item Scanning
-	Positive Korrelation – langsame Geschwindigkeiten führen zu einer besseren Bewertung der Effizienz – warum könnte das so sein?
-	Beim Cartesian Scanning nur eine leichte Korrelation, hier negativ (wie es auch zu erwarten wäre), d.h. schnellere Geschwindigkeit führt tendenziell zu besserer Bewertung der Effizienz 
-	Beide Korrelationen nicht statistisch signifikant (aber auch gerade Anzahl an Testpersonen, daher erwartbar dass eine Signifikanz nicht erreicht wird)

Motion Sickness
Tritt Motion Sickness auf? Wenn ja, wie stark sind die Symptome ausgeprägt?
-	Mittlerer SSQ-Score beim Item etwas höher (Item: 12,155 Cartesian: 10,051) – das bedeutet es traten im Mittel signifikante Symptome auf 
-	bei jeweils 6 Personen bei beiden Verfahren vernachlässigbare Symptome (Score <5) 
-	Höhste erreichte Wert bei Cartesian deutlich höher als bei Item 
-	Bei Item 8 Personen einen Wert >10 (Signifikate Symptome), bei Cartesain 6 
-	Bei Cartesain mehr als die Hälfte der Testpersonen (10) einen Score <10 – das entspricht vernachlässigbaren bis minimalen Symptomen, 3 Personen hingegen einen Score >20, was als besorgniserregend bis problematisch eingestuft wird 
-	Beim Item 8 Personen einen Score <10, 5 Personen einen Score >20 
-	Also mehr höhere Scores beim Item Scanning 
Welche Symptome treten auf vermehrt auf? Gibt es Unterschiede zwischen den Verfahren in Bezug auf die Häufigkeit und Intensität der Symptome?
-	beim Item Scanning die Symptome angestrengte Augen (11 Personen, davon 4 mit moderater Ausprägung) sowie Ermüdung, Kopfdruck und verschwommenes Sehen (6 Personen, keine moderaten Ausprägungen) am häufigsten auftraten
-	am häufigsten wurden angestrengte Augen (11 Personen, davon 2 mit moderater Ausprägung) und Ermüdung (8 Personen, keine moderaten Ausprägungen) berichtet. Im Vergleich zum Item Scanning traten verschwommenes Sehen und Kopfdruck etwas seltener auf (3 bzw. 4 Personen)
-	das heißt beim Item Scanning sind diese Symptome tendenziell bei mehr Personen aufgetreten und es gibt auch öfter moderate Ausprägungen 
-	bei Cartesain trat etwas häufiger leichte Ermüdung auf  
-	hinsichtlich der Kategorien traten die wenigsten Symptome bei Nausea auf (vernachlässigbare bis minimale Symptome), die meisten bei Oculomotor (Signifikate bis besorgniserregende Symptome) – gilt für beide Verfahren 
-	bei Oculomotor beim Cartesian die meisten Personen (12) noch Score <20, bei Item nur 8; aber bei Cartesian peak bei 15-20 (5 personen, entspricht besorgniserregenden Symptomen) und nur 3 Personen mit Score <5 (also keine Symptome); Bei Item hingegen 5 Personen ohne Symptome (peak) 
-	In der Kategorie Disorientation bei Item im Mittel höhere Scores 
-	Bei Disorientation bei beiden Verfahren die größte Streuung – 8 Personen bei Item und 9 bei Cartesian mit einem Score <5 (vernachlässigbare Symptome), dafür aber wenige Personen, die sehr hohe Scores erreichen (Item 3 Personen >40, Cartesian 2 Personen über 40) 
-	Es sollte auf jeden Fall überlegt werden, wie das Auftreten der Symptome weiter verringert werden kann (insbesondere bzgl. Oculomotor) 
-	Angestrengte Augen, Ermüdung, Konzentrationsschwierigkeiten könnte auch durch den Aufbau der Evaluation begünstig worden sein – sehr viele neue Informationen auf einmal (neue Steuerung lernen), viel Zeit in Anspruch genommen (45-60 Minuten) – es ist einfach anstrengend sich über so einen Zeitraum zu konzentrieren – beide Verfahren getestet (doppelte Aufnahme von Inhalten und Anzahl an Aufgaben)
-	Es sollte dennoch überlegt werden, wie die Verfahren optimiert werden können, um die Symptome zu vermeiden/zu verringern 
Bestehen Korrelationen zwischen dem Auftreten von Motion Sickness Symptomen und der Bewertung der Usability/UX sowie der Interaktionsgeschwindigkeit? 
-	SUS: Korrelationsanalyse zeigen schwachen positiven Zusammenhang für Item und mäßigen negativen Zusammenhang bei Cartesian – geringerer SSQ-Score begünstigt eine bessere Bewertung der Usability (beides nicht statistisch signifikant) 
-	Hierbei ist zu berücksichtigen, dass der SUS bereits nach dem technischen Abschnitt ausgefüllt wurde und der SSQ am Ende – d.h. dass ggf. Symptome, die erst im Verlauf des inhaltsbasierten Abschnitts aufgetreten sind (bspw. aufgrund der längeren Nutzungszeit o.Ä.) nicht in die Bewertung der SUS eingeflossen sind 
-	Es könnte also z.b. sein, dass beim Item Scanning kein stärkerer Zusammenhang erkennbar ist, weil die Symptome erst später aufgetreten sind, wohingegen sie beim Cartesain früher auftraten und so die Bewertung der SUS stärker beeinflussten (das ist aber nur eine Möglichkeit/Vermutung, kann nicht sicher geprüft werden)
-	UEQ Attraktivität: leichte positive Korrelation beim Item, leichte negative bei Cartesain (beides nicht signifikant)
-	Bei Cartesain: niedriger SSQ führt zu höherer Attraktivität (wäre nachvollziehbar dass das Verfahren eher als angenehm, gut und sympathisch bewertet wird, wenn keine bis wenige Symptome auftreten
-	Zusammenhang allerdings nur sehr gering und praktisch dementsprechend unbedeutend – also hat das keinen wirklichen Einfluss 
-	Bei Item ist die positive Korrelation spannend, mehr Symptome führen zu einer besseren Bewertung – Warum könnte das so sein?
-	UEQ Durchschaubarkeit: bei beiden Verfahren eine negative Korrelation, bei Item schwach bei Cartesian stark und sogar statistisch signifikant 
-	Also tendenziell Zusammenhang: weniger Symptome bessere Bewertung der Durchschaubarkeit – das deutet darauf hin, dass weniger Symptome bei einer besseren Durchschaubarkeit entstehen – also wenn das Verfahren einfach, übersichtlich und verständlich ist, treten unter Umständen weniger Symptome auf 
-	Interaktionsgeschwindigkeit: positive Korrelation bei beiden Verfahren (bei Item mäßig, bei Cartesain stark und sogar statistisch signifikant) 
-	Schnellere Interaktionsgeschwindigkeiten stehen in Zusammenhang mit niedrigem SSQ-Score 
-	Option 1: Wenn die Personen weniger Symptome verspüren können sie schneller Interaktionen durchführen (das Auftreten von Motion Sickness könnte dazu führen, dass Interaktionen langsamer durchgeführt werden) 
-	Option 2: Wenn Personen schneller Interaktionen durchführen können, treten weniger Symptome auf 
-	Da auf die Interaktionsgeschwindigkeit beim Item Scanning weniger Einfluss seitens der Nutzenden genommen werden kann, liegt es nahe, dass die Korrelation hier weniger deutlich ist als beim Cartesian 


Weiteres abseits der Fragestellungen 
-	Die meisten Interaktionen finden beim Cartesian in der Mitte statt (Ausrichtung des gewünschten Objekts in der Mitte des Sichtfelds) --> Überlegung die Bewegung des Scans so zu gestalten, dass dieser die Mitte schneller erreicht. Dadurch könnte die Geschwindigkeit optimiert werden 
-	Scan Rate sollte auf jeden Fall einstellbar gemacht werden 
-	Einsatz von Kopfbewegungen beim Cartesain führt zu mehr Kontrolle über das System – Fehler können leichter korrigiert werden und die Geschwindigkeit kann erhöht werden – Nutzende haben die mehr Möglichkeiten das Verfahren individuell zu nutzen (je nachdem wie groß der Bewegungsradius des Kopfes ist)

Außerdem (auf alle Ergebnisse bezogen):
-	Nur recht kleine Anzahl an Teilnehmenden, daher können die Ergebnisse zu Tendenzen aufzeigen 
-	Statistische Signifikanz ist bei der kleinen Anzahl an Testpersonen nicht so leicht zu erreichen 

Zusammenfassung 
Was sind Stärken der Verfahren
Wo liegt verbesserungspotenzial 

Ist ein Verfahren geeigneter für VR? (FF3) 
Ableitung von Empfehlungen (FF4) 
Inwiefern wurden die Zielsetzungen der Arbeit erfüllt 


\section{Ausblick}

Manchmal konnte im Rahmen einer Bachelor- oder Masterarbeit ein
wissenschaftliches Thema auch so tief bearbeitet werden, dass wir es auf einem
wissenschaftlichen Workshop oder einer Konferenz veröffentlichen konnten.

Für Themen aus dem Bereich Virtuelle oder Erweiterte Realität bietet sich dazu
zum Beispiel der jährlich stattfindende Workshop der Fachgruppe VR/AR der
Gesellschaft für Informatik an. \citet{Bluhm:Sonar:2009} ist nur ein Beispiel
von mehreren Veröffentlichungen aus unserer Gruppe.



