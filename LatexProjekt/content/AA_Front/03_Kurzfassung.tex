\chapter{Kurzfassung}

Diese Arbeit untersucht die Entwicklung binärer Interaktionsschnittstellen für Virtual Reality (VR) Anwendungen mit dem Ziel, die Zugänglichkeit für Menschen mit motorischen Beeinträchtigungen zu verbessern. Auf Basis einer strukturierten Konzeption wurden zwei Ansätze entwickelt, die auf den etablierten Scanning-Verfahren Automatic Item Scanning und Continuous Cartesian Scanning basieren und gezielt für den Einsatz in VR adaptiert wurden. Beide Ansätze wurden prototypisch auf Basis des Tools PaneoVR implementiert.

Die anschließende Evaluation der implementierten Schnittstellen untersuchte die Unterschiede hinsichtlich Effizienz, Erlernbarkeit, Robustheit, Usability, User Experience (UX) und dem Auftreten von Motion Sickness. Die Ergebnisse zeigen, dass das Item Scanning durch technische Vorteile wie eine höhere Interaktionsgeschwindigkeit und Robustheit überzeugt, während das Cartesian Scanning in den subjektiven Bewertungen, insbesondere in Bezug auf Usability und UX, positiver wahrgenommen wird.
Beide Ansätze zeigen spezifische Stärken und Schwächen. Auf Basis der Evaluation wurden konkrete Verbesserungsansätze abgeleitet und Empfehlungen für die Entwicklung binärer Interaktionsschnittstellen in VR-Anwendungen formuliert. Zu diesen Empfehlungen gehören u. a. die Implementierung unterschiedlicher Scanning-Verfahren, die Integration effizienter Mechanismen zur Fehlerkorrektur sowie die ergonomische Platzierung von UI- und Interaktionselementen.
Eine zentrale Limitation der Ergebnisse besteht in der fehlenden zielgruppenspezifischen Evaluation, wodurch die Übertragbarkeit der Ergebnisse auf Menschen mit motorischen Beeinträchtigungen eingeschränkt ist. Zukünftige Arbeiten sollten die identifizierten Verbesserungsansätze umsetzen und durch zielgruppenspezifische Untersuchungen validieren. Die gewonnenen Erkenntnisse leisten einen Beitrag zur Entwicklung inklusiver VR-Anwendungen und tragen zur Verbesserung der Barrierefreiheit für Menschen mit motorischen Beeinträchtigungen bei.