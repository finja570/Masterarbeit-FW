\chapter{Abstract}

This thesis examines the development of binary interaction interfaces for virtual reality (VR) applications with the aim of improving accessibility for individuals with motor impairments. Based on a structured conceptualization, two approaches were developed that build on the established scanning methods Automatic Item Scanning and Continuous Cartesian Scanning, which were specifically adapted for use in VR. The implementation of both approaches was prototypically conducted on the basis of the tool PaneoVR.

The subsequent evaluation of the implemented interfaces analyzed differences in terms of efficiency, learnability, robustness, usability, user experience (UX), and the occurrence of motion sickness. The findings demonstrate that Item Scanning offers technical advantages such as higher interaction speed and robustness, while Cartesian Scanning is perceived more favorably in subjective assessments, particularly regarding usability and UX.
Both approaches exhibit distinct strengths and weaknesses. Based on the evaluation, specific improvement strategies were derived, and recommendations for the development of binary interaction interfaces in VR applications were formulated. These recommendations include, among others, the implementation of diverse scanning methods, the integration of efficient error-correction mechanisms, and the ergonomic placement of UI and interaction elements.
A key limitation of the results lies in the lack of a target group-specific evaluation, which restricts the generalizability of the findings to individuals with motor impairments. It is therefore recommended that future research implement the identified improvement strategies and validate them through evaluations conducted with the target population. The insights gained contribute significantly to the development of inclusive VR applications and advance the accessibility of virtual environments for individuals with motor impairments.
