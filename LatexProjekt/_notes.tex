% Alles wichtige zum Layout und Design etc. ist grundlegend eigentlich in den tex-Dateien im Wurzel-Verzeichnis angesiedelt. Unter content sind die inhaltlichen Sachen zu Finden.

% Von der voherigen Vorlage sind alle Texte übernommen worden.
% Ganz am Ende von Kapitel 6 sind noch einige nützliche Dinge (Templates, Beispiele etc.) ergänzt - wie eine Methode zum Einfügen von Bildern, Tabellen, das Glossar etc.
% Das steht lose am Ende und ist (noch) nicht in den Text eingearbeitet, was man irgendwann noch machen könnte.

% Die Verzeichnisstruktur ist auch ähnlich zur Vorlage, aber nicht 1:1 übernommen, die könnte jedoch bei Bedarf noch verbessern.

% Vom LaTeX-Code ist nicht so viel übernommen, der ist im Grunde neu erstellt.

% Als Basis bei LaTeX gibt man immer eine documentclass an (steht in der main.tex, relativ weit oben).
% Es gibt einige eingebaute (wie article, book entsprechend für Artikel und Bücher).
% Ich habe mich als documentclass für scrbook entschieden. Das ist eine Klasse, die kommt aus dem sogenannten KOMA-Script, das ist nicht im Standard-LaTeX drin, sondern ist eher eine "erweiterte Standardklasse", die von einer Privatperson entwickelt wird. Da sind viele Dinge bereits enthalten, die man normalerweise mit eigenen Packages eingebaut werden müsste. Ebenfalls wird scrbook von vielen anderen Abschlussarbeiten verwendet und ist somit insgesamt recht etabliert.
% Es gibt auch noch scrreprt, also für report, das kommt auch aus diesem KOMA-Script. Da scheiden sich so ein bisschen die Geister, ob die Masterarbeit scrbook oder scrreprt nutzt, die haben beide Vor- und Nachteile, da könnte man nochmal schauen, was "besser" für unsere Anwendungsfälle ist, aber das wäre relativ schnell geändert.

% Folgende Sachen können noch in Zukunft da verbessert werden:
% - Der Quelltext ist noch nicht wirklich schön, man könnte da noch überall Kommentare ergänzen und das ganze aufräumen.
% - Man könnte eventuell eine andere Schriftart wählen als diese standardmäßige von LaTeX (vielleicht ist da eine andere besser lesbar oder so, aber das wäre ja grundsätzlich schnell umzusetzen).
% - Das originale Template von Thies hat relativ viele Pakete eingebunden, ich habe jetzt nur die eingebunden, die notwendig sind, damit das Template so läuft (und möglichst "lightweight" bleibt). Man könnte also nochmal überprüfen, welche man zusätzlich übernehmen könnte / welche nützlich sein könnten (z. B. gibt es noch welche für irgendwelche Diagramme und Quellcodehighlighting etc. Da könnte man mal schauen, was da noch gut zu gebrauchen ist).

% Ansonsten kann das Design, bei Bedarf, noch geändert / verbessert werden. Das jetzige entspricht (bis auf Dinge wie Abstände etc.) grundlegend dem vorherigen Template.
